\section{Appendix B}
\label{sec:appendix:b}

\subsection{Furture Work}
\label{sec:futurework}
A few natural  open questions are the following.
The first 2 are from Section \ref{sec:recomb}.

\begin{openproblem}
    Combinatorial rigidity for
    periodic structures is an active area of research. This paper motivates
    a study of the rigidity of self-similar structures,
    with self-similar groups replacing periodic groups.
\end{openproblem}

\begin{openproblem}
    The pinned line incidence structures of Section \ref{sec:pinnedline},
    for example in the case of collagen microfibrils, whose function is
    elastic contraction, should be considered
    congruent under projective transformations. I.e, the projective group
    should be factored out as a trivial motion in a new project for
    extending the combinatorial rigidity characterization of such systems.
    (Currently we permit no trivial motions at all).
\end{openproblem}

\begin{openproblem}
What is the complexity of the restricted OMD (optimal modification for
decomposition) problem?
This has the potential to be difficult. For example, when the
isostatic completion is required to be a 2-tree the restricted OMD
problem is reducible to the maximum spanning series-parallel subgraph
problem shown by \cite{cai1993spanning} to be NP-complete even if the input
graph is planar of maximum degree at most 6. However, since the OMD
problem has other input restrictions such as not having any proper
isostatic subgraphs, it is not clear if the reverse reduction exists
and hence it is unclear whether the OMD problem is NP-complete.

The same holds for the restricted OMD problem where the isostatic
completion is required to be a tree-decomposable graph of low
Cayley complexity (i.e, have special, small DR-plans). One potential
obstacle to an indecomposable graph $G$'s membership in the restricted
OMD$_k$ for small $k$ is if $G$ is tri-connected and has large girth.
In fact, 6-connected (hence rigid) graphs with arbitrarily large girth
have been constructed in \cite{servatius2000rigidity}.
\end{openproblem}

The next is the reverse direction of Observation \ref{obs:OC_to_OMD} in Section \ref{sec:table}.

\begin{openproblem}
    Is the OMD (optimal modification for decomposition) problem
    reducible to the OC (Optimal completion) problem?
\end{openproblem}

The last problem is from
Section \ref{sec:bodypin}.

\begin{openproblem}
    What is the complexity of the optimal completion problem when the given
    graph has more than 2-dofs?
    Our proof for the 1 and 2-dof cases relied heavily on the matroidal
    properties of their corresponding $(k,l)$-tightness.
    For higher number of dofs, the $(k,l)$ characterization is
    no longer matroidal \cite{Lee:2007:PGA}.
    As a result, the major obstacle is that
    there is no easy way of obtaining an optimal or canonical
    $k$-dof DR-plan in general.
    Even assuming such  a DR-plan is available,
    if higher dofs had the same characteristics,
    Observation \ref{obs:algebraic_completion}
    raises questions about the correct measure of DR-plan size that
    captures algebraic complexity for recombining graphs with many dofs
    (this is not an issue in the isostatic case). Unless some restrictions
    can be found and taken advantage of, the $k$-dof optimal
    completion problem would  have complexity exponential in $k$.
\end{openproblem}




\subsection{Open Problems}

\todo{added these in newest conclusion}

A few natural  open questions concern the following common theme
that runs through the paper:

\begin{openproblem}
For fixed $k$ we have polynomial time optimal DR-planning,
recombination (modification) in the presence of $k$ overconstraints (Section
\ref{sec:DRP},
optimal modification for decomposition {\sl OMD}$_k(G)$,
when at most $k$ constraints are removed
\ref{sec:recomb}
and
also optimal completion
using at most $k\le 2$ constraints in the body-pin and
triangle-multipin cases
under a modified measure for optimizing the
DR-plan \ref{sec:table}.
However, $k$ appears in the exponent for
the straightforward algorithms for these problems.
\end{openproblem}

One problem in the above theme is from
Section \ref{sec:bodypin}.
\begin{openproblem}
    What is the complexity of the optimal completion problem when the given
    graph has more than 2-dofs?
    Our proof for the 1 and 2-dof cases relied heavily on the matroidal
    properties of their corresponding $(k,l)$-tightness.
    For higher number of dofs, the $(k,l)$ characterization is
    no longer matroidal \cite{Lee:2007:PGA}.
    As a result, the major obstacle is that
    there is no easy way of obtaining an optimal or canonical
    $k$-dof DR-plan in general.
    Even assuming such  a DR-plan is available,
    if higher dofs had the same characteristics,
    Observation \ref{obs:algebraic_completion}
    raises questions about the correct measure of DR-plan size that
    captures algebraic complexity for recombining graphs with many dofs
    (this is not an issue in the isostatic case). Unless some restrictions
    can be found and taken advantage of, the $k$-dof optimal
    completion problem would  have complexity exponential in $k$.
\end{openproblem}

Another problem from the above theme is from Section \ref{sec:recomb}
\begin{openproblem}
What is the complexity of the restricted OMD (optimal modification for
decomposition) problem?
This has the potential to be difficult. For example, when the
isostatic completion is required to be a 2-tree the restricted OMD
problem is reducible to the maximum spanning series-parallel subgraph
problem shown by \cite{cai1993spanning} to be NP-complete even if the input
graph is planar of maximum degree at most 6. However, since the OMD
problem has other input restrictions such as not having any proper
isostatic subgraphs, it is not clear if the reverse reduction exists
and hence it is unclear whether the OMD problem is NP-complete.

The same holds for the restricted OMD problem where the isostatic
completion is required to be a tree-decomposable graph of low
Cayley complexity (i.e, have special, small DR-plans). One potential
obstacle to an indecomposable graph $G$'s membership in the restricted
OMD$_k$ for small $k$ is if $G$ is tri-connected and has large girth.
In fact, 6-connected (hence rigid) graphs with arbitrarily large girth
have been constructed in \cite{servatius2000rigidity}.
\end{openproblem}

The next is the reverse direction of Observation \ref{obs:OC_to_OMD} in
Section \ref{sec:table}.

\begin{openproblem}
    Is the OMD (optimal modification for decomposition) problem
    reducible to the OC (Optimal completion) problem?
\end{openproblem}

More general problem directions are the following.
\begin{openproblem}
    Combinatorial rigidity for
    periodic structures is an active area of research. This paper motivates
    a study of the rigidity of self-similar structures,
    with self-similar groups replacing periodic groups.
\end{openproblem}

\begin{openproblem}
    The pinned line incidence structures of Section \ref{sec:pinnedline},
    for example in the case of collagen microfibrils, whose function is
    elastic contraction, should be considered
    congruent under projective transformations. I.e, the projective group
    should be factored out as a trivial motion in a new project for
    extending the combinatorial rigidity characterization of such systems.
    (Currently we permit no trivial motions at all).
\end{openproblem}






