\section{Using the canonical, optimal DR-plan for Realizing (Solving) Qusecs}
\label{sec:recomb}



\begin{figure*}\centering
\begin{subfigure}{.3\linewidth}\centering
  \begin{tikzpicture}[scale=2]
        \tikzstyle{v}=[draw, circle, minimum size=0.1cm, font=\footnotesize]
        \tikzstyle{c}=[draw, circle, inner sep=2, fill=black]
        \tikzstyle{e}=[]

        \node[c] (v0) at (  -1,  0)     [label=left:$v_0$]{};
        \node[c] (v1) at (-0.5,  0.866) [label=above left:$v_1$]{};
        \node[c] (v2) at ( 0.5,  0.866) [label=above right:$v_2$]{};
        \node[c] (v3) at (   1,  0)     [label=right:$v_3$]{};
        \node[c] (v4) at ( 0.5, -0.866) [label=below right:$v_4$]{};
        \node[c] (v5) at (-0.5, -0.866) [label=below left:$v_5$]{};

        \tedge{v3}{v2}{solid}{}{};
        \tedge{v2}{v1}{solid}{}{};
        \tedge{v1}{v0}{solid}{}{};
        \tedge{v0}{v5}{solid}{}{};
        \tedge{v5}{v4}{solid}{}{};
        \tedge{v4}{v3}{solid}{}{};

        \tedge{v3}{v0}{solid}{}{};
        \tedge{v2}{v5}{solid}{$e$}{left, pos=0.15};
        \tedge{v1}{v4}{solid}{}{};
    \end{tikzpicture}

    \caption{}
\end{subfigure}
%
\begin{subfigure}{.3\linewidth}\centering
  \begin{tikzpicture}[scale=2]
        \tikzstyle{v}=[draw, circle, minimum size=0.1cm, font=\footnotesize]
        \tikzstyle{c}=[draw, circle, inner sep=2, fill=black]
        \tikzstyle{e}=[]

        \node[c] (v0) at (0,  1) [label=above:$v_0$]{};
        \node[c] (v1) at (0,  0)     [label=below:$v_1$]{};
        \node[c] (v2) at (-0.5, 1.5) [label=above:$v_2$]{};
        \node[c] (v3) at (1, 1)     [label=above:$v_3$]{};
        \node[c] (v4) at (1, 0) [label=below:$v_4$]{};
        \node[c] (v5) at (1.5,  1.5) [label=above:$v_5$]{};

        \tedge{v3}{v2}{solid}{}{};
        \tedge{v2}{v1}{solid}{}{};
        \tedge{v1}{v0}{solid}{}{};
        \tedge{v0}{v5}{solid}{}{};
        \tedge{v5}{v4}{solid}{}{};
        \tedge{v4}{v3}{solid}{}{};

        \tedge{v3}{v0}{solid}{}{};
        \tedge{v2}{v5}{dashed}{$e$}{above};
        \tedge{v1}{v4}{solid}{}{};
    \end{tikzpicture}

    \caption{}
\end{subfigure}
%
\begin{subfigure}{.3\linewidth}\centering
  \begin{tikzpicture}[scale=2]
        \tikzstyle{v}=[draw, circle, minimum size=0.1cm, font=\footnotesize]
        \tikzstyle{c}=[draw, circle, inner sep=2, fill=black]
        \tikzstyle{e}=[]

        \node[c] (v0) at (0,  1) [label=above:$v_0$]{};
        \node[c] (v1) at (0,  0)     [label=below:$v_1$]{};
        \node[c] (v2) at (-0.5, 1.5) [label=above:$v_2$]{};
        \node[c] (v3) at (1, 1)     [label=above:$v_3$]{};
        \node[c] (v4) at (1, 0) [label=below:$v_4$]{};
        \node[c] (v5) at (1.5,  1.5) [label=above:$v_5$]{};

        \tedge{v3}{v2}{solid}{}{};
        \tedge{v2}{v1}{solid}{}{};
        \tedge{v1}{v0}{solid}{}{};
        \tedge{v0}{v5}{solid}{}{};
        \tedge{v5}{v4}{solid}{}{};
        \tedge{v4}{v3}{solid}{}{};

        \tedge{v3}{v0}{solid}{}{};
        \tedge{v2}{v5}{dashed}{$e$}{above};
        \tedge{v1}{v4}{solid}{}{};

        \tedge{v1}{v3}{solid, ultra thick}{$f$}{left, pos=0.4};
    \end{tikzpicture}

    \caption{}
\end{subfigure}

\caption{(a) The $K3,3$ with one labeled edge $e$. (b) The $K3,3$ with $e$ removed (dashed line) and rearranged to illustrate that it is now a partial 2-tree. (c) The $K3,3$ with $e$ removed and $f$ (bold line) added to make a 2-tree. Note that the non-edge $(v_0,v_4)$ would also be a valid choice.}
\label{fig:omd_k33_example}
\end{figure*}


In this section, we consider the {\em optimal recombination}
problem of combining specific solutions of
subsystems in a DR-plan into a solution of their parent system $I$, (without
loss of generality, at the top level of the DR-plan). In the case of isostatic
qusecs, the parent system $I$ is isostatic (the root of the DR-plan), and we
seek  {\em solution(s) (among a finite large number of solutions)
with a specific orientation or chirality}.
In the case of underconstrained qusecs the subsystems are the multiple roots
of the DR-plan, the parent system $I$ is underconstrained, and we typically seek
an efficient algorithmic description of {\em connected component(s) of solutions with a specific orientation or
chirality}.

The main barrier in recombination when given an optimal DR-plan (of smallest possible
size or max fan-in),  is that the number of children (resp. number of
roots) - and correspondingly the  size and complexity of the (indecomposable) algebraic
system $I$ to be solved -
could be arbitrarily large as a function of the size of the input system.
This difficulty can persist even after optimal parametrization of the
indecomposable system $I$ as in \cite{XX}
to minimize its algebraic complexity.

In the following, we formulate the problem of {\em optimal
modification},
of the indecomposable
algebraic system $I$ into a decomposable system with a small DR-plan (low
algebraic complexity).
Leveraging recent results on {\em Cayley
configuration spaces}, our approach to the optimal modification problem
achieves the following: {\em (a) Small DR-plan.} We obtain a  parameterized family of systems
$I_{\lambda_F}$ -  one for each value $\lambda_F$ for the parameters $F$ ,  all of which have
small DR-plans. Thus, given a value $v$ for $\lambda_F$, the system $I_v$ can
potentially be solved or realized easily once the orientation type of the solution
is known  (when the DR-plan size is small enough).
{\em (b) Solution preservation.} Moreover, the union of solution spaces of the systems in the family
$I_{\lambda_F}$
is guaranteed to contain
all of $I$'s solutions. {\em (c) Efficient search.} Finally, this union can be projected into the
so-called {\em Cayley} or distance parameter space $\lambda_F$ that is convex
or otherwise easy to traverse in order to search for $I$'s solution
(or connected component)
of the desired orientation type.
For the case when the modification is bounded, this approach provides an efficient
algorithm for recombination. We first define the decision version of the
problem of optimal modification
for decomposition. The standard optimization versions are straightforward.

\medskip\noindent{\bf Optimal Modification for Decomposition (OMD) Problem.} Given a generically
independent graph $G = (V,E)$
with no non-trivial proper isostatic subgraph (indecomposable), and 2 constants $k$ and $s$,
does there exist a set of at most $k$  edges $E_1$ and a set of nonedges $F$
such that the modified graph $H = (V, E\setminus E_1 \cup F)$ has a DR-plan
of size at most $s$?  The OMD$_k$ problem is  OMD where $k$ is a fixed bound (not
part of the input).   We say that such a tuple $(G,s)$ is a member of the
set OMD$_k$. We loosely refer to graphs $G$ as OMD with appropriately small $k,s$ or OMD$_k$ graphs
with appropriately small $s$.


It is immediately clear that indecomposable graphs $G$ that belong in OMD$_k$ for small $k$ and
$s$  lend themselves to modification  into decomposable graphs
satisfying Criteria (a) and (b) above.
However, it is not clear how Criterion (c) is met by OMD graphs.
%
\subsection{Using Convex Cayley Configuration Spaces}
%
\ref{2-tree-reduction}
Next we provide the necessary background to describe a specific approach for achieving the requirements
(a)-(c) mentioned above, by restricting the class of reduced graphs $G' =
G\setminus E_1$ and their isostatic completions $H$ in the above definition of the
OMD problem, and using a key theorem of Convex Cayley configuration
spaces \cite{XX}. This theorem characterizes the class of graphs $H$ and non-edges $F$ (Cayley
parameters), such that the set of vectors $\lambda_F$ of  attainable lengths of
the nonedges $F$
is always convex for any
given lengths $\delta$ for the edges of $H$ (i.e, over all the realization
s of the bar-joint constraint system or linkage $(H,\delta)$ in 2 dimensions).
This set is called the (2-dimensional) {\em Cayley configuration space}
of the linkage $(H,\delta)$ on the Cayley parameters $F$, denoted
$\Phi_F(H,\delta)$ and can be viewed as a ``projection'' of the cartesian
realization space of $(H,\delta)$ on the Cayley parameters $F$.
Such graphs $H$ are said to have {\em convexifiable Cayley configuration spaces
with parameters $F$} (short: {\em convexifiable}). To state the theorem, we first have to define the
notion of {\em 2-sums} and {\em 2-trees}.
Let $H_1$ and $H_2$ be two graphs on disjoint sets of vertices $V_1$ and
$V_2$, with edge sets $E_1$ and $E_2$ containing edges $(u,v)$ and $(w,x)$
respectively.  A {\em 2-sum} of
$H_1$ and $H_2$ is a
graph $H$  obtained by taking the union of $H_1$ and $H_2$ and identifying $u=w$ and $v=w$.
In this case, $H_1$ and $H_2$ are called {\em 2-sum components} of $H$.
A {\em minimal 2-sum component} of $H$ is  one that cannot be further split
into 2-sum components.
A {\em 2-tree} is recursively obtained by taking
a 2-sum of two 2-trees, with the base case of 2-tree being a triangle.
A {\em partial 2-tree} is a 2-tree minus some edges.
Partial 2-trees are characterized as having $K_4$ as forbidden minor are also called series parallel graphs.

\begin{theorem}
    \label{convexcayley}
    \cite{XX}
    $H$ has a convexifiable Cayley configuration space  with parameters $F$
    if and only if for each $f\in F$  all the minimal 2-sum components
    of $H\cup F$ that contain both endpoints of $f$ are partial 2-trees.
    The Cayley configuration space $\Phi_F(H,\delta)$
    of a bar-joint system or linkage $(H,\delta)$ is a convex polytope.
    When $H\cup F$ is a 2-tree, the
    bounding hyperplanes of this polytope are triangle inequalities
    relating the lengths of edges of the triangles in $H\cup F$.
\end{theorem}

The idea of our approach to achieve the criteria
(a)-(c) begins with the following simple but useful theorem.

\begin{theorem}
    \label{omdk}
Given an indecomposable graph $G$,
let $G'$ be a spanning partial 2-tree subgraph in $G$ with
$k$ fewer edges than $G$.
Then  $(G,2)$ belongs in the set OMD$_k$.
\end{theorem}

\begin{proof}
The proof follows from the fact that 2-trees are well decomposable and have
simple DR-plans of size 2.
We know that $G$ can be reduced
by removing $k$ edges to create a partial-2-tree $G'$
which can then be completed to an (isostatic) 2-tree by adding some set of non-edges
$F$. Thus the modified graph $H = G'\cup F$ has  a DR-plan of size 2, proving the theorem.
\end{proof}.

We refer to such graphs $G$ in short as {\em $k$-approximately
convexifiable,} where the reduced graphs $G'$ and isostatic completions $H$
are convexifiable.
As observed earlier, since graphs such as $G$ are in OMD$_k$, Criteria (a) and (b)
are automatically met for small enough $k$.
Criterion $c$ is addressed
as described in the following efficient search procedure
which clarifies the dependence of the complexity on the number and ranges  of Cayley
parameters $F$.

\begin{theorem}
    \label{criterionc}
    [Efficient search]
    For an indecomposable, $k$-approximately convexifiable graph $G =
    (V,E)$,
   let $G' = (V,E' =E\setminus D)$ be a spanning partial 2-tree subgraph
    where $|D| \le  k$. Let  $F$ be a set of nonedges of $G$ such that
    $H = (V, E'\cup F)$ is a 2-tree.
    Each solution $p$ (or connected component of a solution space)
    of $(G,\delta)$
    of an orientation type $\sigma_p$ can be found in time $O(\log(W))$ where
    $W$ is the number of cells of desired accuracy (discrete volume) of
    the convex
    polytope $\Phi_F(G',\delta_E')$.  The (discrete) volume $W$ is exponential in
    $|F|$ and polynomial in the (discrete range) of the parameters in $F$.
\end{theorem}

\begin{proof}
    The cartesian realization
    space of $(H,<\delta_E', \lambda_F>)$ is
    computed easily with a DR-plan of size 2, and is the union of $2^t$
    solutions (modulo orientation preserving isometries) each with a distinct orientation type, where $t$ is the number of triangles in the 2-tree
    $H$;
    here $\delta_{S}$ is the restriction of the length vector $\delta$ to
    the edges in $S$.
    A desired solution $p$ (or connected component of a solution space)
    of $(G,\delta)$
    of an orientation type $\sigma_p$ can be found
    by a subdivided binary search   of the cartesian realization space of
    $(H, <\delta_{E'},\lambda_F>)$ of orientation type $\sigma_p$ -
    as $\lambda_F$ ranges over
    the discretized convex
    polytope $\Phi_F(G',\delta_E')$ with bounding hyperplanes described in
    Theorem \ref{convexcayley}.
    A solution $p$  is found  when the lengths for nonedges in $D$ match
    $\delta_D$.
\end{proof}



\medskip\noindent{\bf Example}.
A graph $G=K_{3,3}$  cannot be decomposed into any nontrivial isostatic
graphs, i.e, its DR-plan has a root and 9
children corresponding to the 9 edges. Solving or recombining the system
$(G,\delta)$
corresponding to the root of
this DR-plan involves solving a quadratic system with 8 equations and variables,
Instead of simultaneously solving this system, we could instead use the fact
that $G=K_{3,3}$ is in OMD$_1$: remove the
edge $e$ in Figure \ref{fig:omd_k33_example} to give a partial 2-tree $G'$.
Now add the nonedge $f$ to give a 2-tree $H$ with a DR-plan of size 2.
The Cayley configuration space $\Phi_f(G', \delta_{E\setminus e})$
is a single interval of attainable length values $\lambda_F$ for
the edge $f$.
When $\delta$ is generic, i.e, does not admit collinearities or coincidences
in the realizations of $(G,\delta)$,
the realization space of $(H, <\delta_{E\setminus e}, \lambda_f>)$ has 16
solutions $q_p^{\lambda_f}$ (modulo orientation preserving isometries),
with distinct orientation types $\sigma_p$,
(two orientation choices for each of the 4 triangles)
that can be obtained by solving a sequence of 4 single quadratics in 1
variable (DR-plan of size 2).
By subdivided
binary search
in the interval $\lambda_f \in
\Phi_f(G', \delta_{E\setminus e})$, the desired
solution $p$  of $(G,\delta)$ is found when the length of the nonedge
$e$ in  the realization
$q^p_{\lambda_f}$
is $\delta_e$.
%
\subsection{Optimized Modification by Enlarging the class of Reduced
Graphs}
\label{tdecomp}
It is possible in principle to decrease $k$ for a OMD$_k$ graph
(i.e, the number of edges to be removed
to ensure an isostatic completion that is decomposable with a small DR-plan) by
considering reduced graphs $G'$ (and modified graphs $H$) that come from a larger class than
partial 2-trees but nevertheless have convex Cayley configuration spaces at
least when the realization space is restricted to a sufficiently
comprehensive orientation type.
In particular, the so-called {\em tree-decomposable graphs of low Cayley
complexity} \cite{XX} include the partial 2-trees and many others that are
not partial 2-trees. These too result in DR-plans of size 2 or 3, putting $G$ in the class
OMD$_k$ and thus meeting Criteria (a) and (b). The Criterion (c) is met
- for example when $k=1$ -  because 1 dof Cayley configuration spaces of
linkages based on such graphs $G'$
are known to be single intervals
when a comprehensive orientation type $\sigma_p$ of the sought solution $p$ is given.
In addition, the bounds of these intervals are of low algebraic
complexity.
More precisely, the bounds  can themselves be computed using a DR-plan of size 2 or 3,
i.e, the computation of these bounds is tree-decomposable. Alternatively,
the bounds are in a simple quadratic or radically solvable extension field over the rationals,
or can be computed by solving a
triangularized system of quadratics.
