\section{Preliminaries}

\subsection{Definitions}

In the context of this paper, a \textbf{graph} $G=(V,E,w_V,w_E)$ is a set of vertices, $V$, and a set of edges, $E$, defined as a tuple of vertices from $V$ (undirected). Additionally, there are two weight functions, $w_V: V \to \mathbb{R}^+$ and $w_E: E \to \mathbb{R}^+$.

The \textbf{density} of graph $G$ is
\begin{equation}
    d(G) = \sum_{e\in E}{w_E(e)} - \sum_{v\in V}{w_V(v)}
\end{equation}

A graph is \textbf{independent} (or \textbf{sparse}) if \todo{define}.

A graph is \textbf{over-constrained} if, given constant $k$, there exists some induced subgraph $S\subseteq G$ such that $d(S) > k$.

A graph is \textbf{well-constrained} if, given constant $k$, $d(G) = k$ and for all induced subgraphs $S\subseteq G$, (1) $d(S)\leq k$, or (2) given a set of trivial graphs $T$, $S$ is isomorphic to one of the graphs in $T$ (i.e.\ $S$ is trivial).

A graph is \textbf{rigid} if, given constant $k$, there exists some spanning subgraph $S\subseteq G$ such that $S$ is well-constrained.

A graph is \textbf{under-constrained} if, given constant $k$, $G$ is not rigid.

A \textbf{trivial} graph is ill-defined in the general case. The only strict requirements are: (1) it must be an over-constrained graph and (2) all of its subgraphs are also trivial.
\todo{Maybe leave the following out until it's needed later?}
In the familiar geometric cases of $d$-dimensional space, all vertex weights are $d$, all edge weights are $1$, and constant $k= -{{d+1}\choose{2}}$. Trivial graphs for 2D would be a single vertex. Trivial graphs in 3D would be a vertex and an edge (2 vertices with an edge between). Etc. These trivial graphs capture the notion of the rotational symmetry that exists in geometric spaces.

The \textbf{decomposition-recombination (DR-) plan} of graph $G$, $DRP(G)$, is defined as a tree that has the following properties:
\begin{enumerate}
    \item The root of the tree `contains' $G$.
    \item For all nodes, $C$, that `contain' nontrivial graphs, its $N$ children, $C_1, \ldots, C_N$, are trivial or well-constrained vertex-maximal proper subgraphs of that node $C$.
    \item The vertex set of $\bigcup_{i=1}^N{C_i}$ is the vertex set of $C$.
    \item A node with a trivial subgraph is a leaf.
\end{enumerate}
% It can also be described recursively as: the root is $G$, its children are the trivial subgraphs and the DR-plans of its well-constrained vertex-maximal proper subgraphs whose union is $G$ itself.

An equivalent DAG can be constructed from this tree such that all nodes containing the same subgraphs are combined into one, with all edges preserved.

Note that nodes will be referred to interchangeably as ``the node that contains the (sub)graph $C$'' and as simply ``$C$''.

The \textbf{complete DR-plan} of graph $G$, $CompleteDRP(G)$, is the unique DR-plan but with a modified rule number 2: the children are \textit{all} of the trivial or well-constrained vertex-maximal proper subgraphs of that node $C$. By this, rule 3 is implicit for a complete DR-plan.

The \textbf{optimal DR-plan} of graph $G$, $OptimalDRP(G)$, is the DR-plan that minimizes the maximum fan-in over all nodes in the tree. It is not necessarily unique.


\subsection{Notation}

$G$ is taken to be a graph with values $(V,E,w_V,w_E)$. To keep problems meaningful we assume that $G$ is connected, as the methods discussed are used to solve systems of equations to establish linkages. It is not useful to solve for two bodies simultaneously if there are no constraints between them. Furthermore, to limit the scope of the paper, $G$ will always be taken to be well-constrained.
% THIS MEANS NO EDGES MAY BE LEFT OUT, ALSO.

$Idc(G,X)$ is the graph induced on $G$ with the vertex set $X\subseteq V$. This can also be overloaded such that, using graph $H=(W,F)$, $Idc(G,H)=Idc(G,W)$.

$C$ is an arbitrary graph in a node in $CompleteDRP(G)$. $C_i$ is the $i^{\text{th}}$ child of $C$. By definition of a DR-plan, it is implied that $C_i$ is a well-constrained vertex-maximal proper subgraph.
