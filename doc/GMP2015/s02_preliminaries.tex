\section{Preliminaries}
\label{sec:prelim}

\subsection{Basics}

In this paper, a \dfn{geometric constraint system} is a multivariate
polynomial (usually bilinear or quadratic) system $G(x,\delta)=0$,
representing constraints with parameters $\delta$ between geometric
primitives  in $\mathbb{R}^2$ represented collectively as $x\in
\mathbb{R}^n$.
%
When the type of constraint (system) is fixed, the system is simply
represented as $(G,\delta)$, where $G$ is the underlying constraint
(hyper)graph $G = (V,E)$ with the vertices $V$ representing the
geometric primitives in $\mathbb{R}^2$ and (hyper)edges $E$
representing the constraints, each with an associated parameter
$\delta$.
%
For example, a \dfn{bar-joint system or linkage} $(G,\delta)$, is a
graph $G=(V,E)$ with fixed length bars as edges, i.e. $\delta: E
\rightarrow \mathbb{R}$; this represents the distance constraint
system $\| x_u -x_v \|_2 = \delta_{u,v}$ for  $(u,v) \in E$, where
$x_u \in \mathbb{R}^2$ represents the coordinates of $u\in V$.

\noindent
\note Geometric constraint systems can also have
inequalities, where the parameters in $\delta$ are small intervals of
values rather than exact values.

In this paper, we will consider 3 types of constraint systems: bar-
joint as well as \dfn{pinned line-incidence}, and \dfn{body-hyperpin}.
In all of these cases, a cartesian \dfn{realization} or \dfn{solution}
$G(p)$ of $(G,\delta)$ is an assignment of coordinates or Euclidean
transformations (poses), $p: V \rightarrow \mathbb{R}^2$ or
$\mathbb{R}^3$, to the vertices of $G$ satisfying the constraints with
parameters $\delta$, modulo orientation preserving isometries
(Euclidean rigid body motions for distance constraint).

Although the realization space itself depends on the constraint
parameters $\delta$, many relevant \dfn{generic} properties of the
constraint system $G(x,\delta)$ are properties of the constraint
(hyper)graph $G$ and do not depend on $\delta$ (or they hold for all
but a measure zero set of $\delta$ values). Many of these are
properties of the jacobian $\Delta_x G(x,\delta)$, often called the
appropriate \dfn{rigidity matrix of $G$} (a matrix of indeterminates).
For example, the \dfn{bar-joint rigidity matrix of the graph $G =
(V,E)$} is a matrix of indeterminates representing the jacobian of the
distance map $\| x_u -x_v \|_2$  for $(u,v) \in E$. The matrix has $2$
columns per vertex in $V$ and one row per edge in $E$, where the row
corresponding to edge $(u,v)$ contains the 2 coordinate indeterminates
for $x_u -x_v$ (resp. $x_v-x_u$) in the 2 columns for $u$ (resp.\
$v$), that is 4 non-zero entries per row.


% XXXXXXXXX
% something has to be said along with asimow roth
% (infinitesimal rigidity implies rigidity) When the rigidity matrix has
% appropriate rank, the realizations or solutions of the corresponding
% constraint system are generically isolated and zero-dimensional if the
% constraint system has equalities and exact values for the parameters
% $\delta$. If the constraint system has inequalities, i.e, if the
% $\delta$ lie in an interval, this claim is approximate: the solutions
% are isolated small, full-dimensional  connected components.

One important property of generic constraint system or (hyper)graph
%that depends on the appropriate rigidity matrix of indeterminates
is \dfn{rigidity} (\note we refer to these as properties of
the constraint system or as properties of the underlying (hyper)graph
interchangeably), i.e., the realizations or solutions of the
corresponding constraint system being generically isolated and zero-
dimensional.%(if the constraint system has equalities and exact values for the parameters $\delta$).
The result by Asimow and Roth \uncited shows a constraint
(hyper)graph is rigid if it is generically \dfn{infinitesimally
rigid}, i.e.\ the number of independent rows of its appropriate
rigidity matrix is at least the number of columns less the number of
rigid body motions, which is 3 for bar-joint systems.
%This condition is called generic {\em infinitesimal rigidity}.

\noindent
\note If the constraint system has inequalities, i.e, if the $\delta$
lie in an interval, the definition of rigidity is approximate: the
solutions are isolated small, full-dimensional  connected components.
% generic rigidity, i.e.\ there exist at most finitely many solutions to the system at generic points,  %to the algebraic system

Other generic constraint system or (hyper)graph properties are the
following:
% (\note we refer to these as properties of the constraint
% system or as properties of the underlying (hyper)graph
% interchangeably).
A constraint (hyper)graph $G$ is \dfn{independent} if its appropriate
rigidity matrix of indeterminates has independent rows (i.e, the
determinant of some square submatrix is not identically zero).
%It is {\em rigid} if the number of independent rows of the rigidity matrix is
%at least the number of columns less the number of rigid body motions,
%which is 3 for distance constraint systems.
It is \dfn{isostatic (minimally rigid, well-constrained)} if it is
both rigid and independent.
%if the number of generically independent rows or the rank of the appropriate rigidity matrix is maximal.
%For example, the maximal rank of a bar-joint rigidity matrix is $2|V| - 3$,
%where $3$ is the number of rotational and translational degrees-of-freedom of a rigid body in $\mathbb{R}^2$.
%The graph $G$ is {\em rigid} if there exists some spanning subgraph $S\subseteq G$ such that $S$ is well-constrained.
It is \dfn{flexible} if it is not rigid, \dfn{under-constrained} if it is independent and not rigid, or \dfn{over-constrained} if it is not independent.


Defining the independence of a subset of edges $E' \subseteq E$ to be
the independence of corresponding rows in the appropriate rigidity
matrix of indeterminates, we obtain the \dfn{rigidity matroid} of a
constraint (hyper)graph $G = (V,E)$.
%The 2-dimensional {\em rigidity matroid} of a constraint (hyper)graph $G = (V,E)$ is a linear matroid  on $E$,
%where a subset of edges $E' \subseteq E$ is {\em independent} in the matroid,
%if the set of corresponding rows in the appropriate rigidity matrix of indeterminates are linearly independent.
There are various results on combinatorial characterization of
independence and rigidity for different types of  rigidity matroid.
For bar-joint rigidity matroid, the famous Laman's theorem \uncited
states that the underlying graph is minimally rigid if and only if
$|E| = 2|V|-3$ and $|E'| \le 2|V'|-3$ for every induced subgraph with
at least 2 vertices. The result by Lovasz and Yemini \uncited shows
that all 6-vertex-connected graphs are rigid in the plane. For bar-
body rigidity matroid, Tay \uncited proved that the underlying
multigraph is minimally rigid if and only if it can be decomposed as
$3$ edge disjoint spanning trees. White and Whiteley \uncited gave
the same characterization using a different technique to study the
algebraic - geomtric conditions called pure condition. Lee, Streinu
and Theran \uncited defined the \dfn{$(k,l)$-sparsity matroid},
where a hypergraph $G$ is called \dfn{$(k,l)$-sparse} if $|E'| \le
k|V'| - l$ for any induced subgraph $(V',E')$ with at least 2
vertices, and \dfn{$(k,l)$-tight} if is $(k,l)$-sparse and $|E| = k|V|
- l$. In general, given a $d$-uniform hypergraph, a $(k,l)$-sparsity
condition is matroidal as long as $l \le dk-1$.



In this paper, a \dfn{qusec} is any \dfn{independent} geometric
constraint system of one of the 3 types mentioned above. In the
remainder of this section and the next two sections~\ref{sec:DRP} and
\ref{sec:recomb} we only consider bar-joint qusecs and graphs.
Relevant formal analogies for the other 2 types of qusecs and
(hyper)graphs are given in the subsequent 2 sections on Applications.



%XXXNow comes the definition of stress vector and flex vector.
%NOTE: something has to be said about what flexes and stresses mean


%TODO
Given a bar-joint graph, a vector (of indeterminates) in the right
null space of its rigidity matrix is called a \dfn{flex vector}. It
has 2 entries per vertex and represents the internal motions of the
system (i.e.\ modulo rigid body motions). For rigid graphs, all flex
vectors are identically zero.
%
A vector (of indeterminates) in the left null space of the rigidity
matrix is called a \dfn{stress vector}. It has one entry per edge and
represents a \dfn{self stress} of the system. For independent graphs,
all stress vectors are identically zero.
%
An \dfn{external stress} $t$ is a $1\times 2|V|$ vector  (of
indeterminates) which specifies a 2D vector acting at each vertex such
that the stress balance equation $sR = t$ holds, where $s$ is a vector
of internal stresses, one for each bar, and $R$ is the rigidity
matrix.












\subsection{Decomposition-Recombination (DR) Plans}

%$G=(V,E,w_V,w_E)$ is a set of vertices, $V$, and a set of edges, $E$, defined as a tuple of vertices from $V$ (undirected). Additionally, there are two weight functions, $w_V: V \to \mathbb{R}^+$ and $w_E: E \to \mathbb{R}^+$. The \textbf{density} of graph $G$ is $d(G) = \sum_{e\in E}{w_E(e)} - \sum_{v\in V}{w_V(v)}$.

%Given a constant $k$, a graph $G$ is:
%\begin{itemize}
%    \item \textbf{independent} (or \textbf{sparse}) if, for all non-trivial subgraphs $S\subseteq G$, $d(S) \leq k$.
%
%    \item \textbf{over-constrained} if it is not independent.
%
%    \item \textbf{well-constrained} (or \textbf{tight}) if $G$ is independent and $d(G)=k$.
%
%    \item \textbf{rigid} if there exists some spanning subgraph $S\subseteq G$ such that $S$ is well-constrained.
%
%    \item \textbf{under-constrained} if $G$ is independent and not rigid.
%
%    % \item \textbf{trivial} if (1) it is over-constrained graph and (2) all of its subgraphs are also trivial. Trivial graphs are input.
%\end{itemize}
%
% \begin{definition}
%     A graph is \textbf{over-constrained} if, given constant $k$, there exists some
%     % induced
%     subgraph $S\subseteq G$ such that $d(S) > k$.
% \end{definition}

% \begin{definition}
%     A graph is \textbf{well-constrained} if, given constant $k$, $d(G) = k$ and for all
%     % induced
%     subgraphs $S\subseteq G$, (1) $d(S)\leq k$, or (2) given a set of trivial graphs $T$, $S$ is isomorphic to one of the graphs in $T$ (i.e.\ $S$ is trivial).
% \end{definition}



% \begin{definition}
%     A \textbf{trivial} graph is ill-defined in the general case. The only strict requirements are: (1) it must be an over-constrained graph and (2) all of its subgraphs are also trivial.
%     \todo{Maybe leave the following out until it's needed later?}
%     In the familiar geometric cases of $d$-dimensional space, all vertex weights are $d$, all edge weights are $1$, and constant $k= -{{d+1}\choose{2}}$. Trivial graphs for 2D would be a single vertex. Trivial graphs in 3D would be a vertex and an edge (2 vertices with an edge between). Etc. These trivial graphs capture the notion of the rotational symmetry that exists in geometric spaces.
% \end{definition}

A \dfn{trivial} graph is a single vertex.
%
\begin{definition}\label{def:drp}
    The \dfn{decomposition-recombination \mbox{(DR-)} plan} of graph
    $G$, \uncited $DRP(G)$, is defined as a forest that has the
    following properties:
    \begin{enumerate}
        \item Each node represents/contains/is a rigid vertex-induced
        subgraph of $G$.
        % \item For a node, $C$, that is  a non-trivial graph, its $N$
        % children, $C_1, \ldots, C_N$, are rigid vertex-maximal proper
        % subgraphs of $C$.
        \item The vertex set of $\bigcup_{i=1}^N{C_i}$ is the vertex
        set of $C$.
        \item A leaf node is a trivial subgraph.
        \item A root node is a maximal rigid subgraph of $G$.
    \end{enumerate}

    % It can also be described recursively as: the root is $G$, its children are the trivial subgraphs and the DR-plans of its well-constrained vertex-maximal proper subgraphs whose union is $G$ itself.

    An equivalent DAG can be constructed from the above forest such
    that all nodes containing the same subgraphs are identified into a
    single node. In this manuscript, however, we retain a forest as
    defined above. When a graph is isostatic, its DR-plan is a tree.

    %Note that nodes will be referred to interchangeably as ``the node that
    %represents or contains the (sub)graph $C$'' and as simply ``$C$''.

    A DR-plan is \dfn{complete} if it satisfies an additional rule:
        \item For a node, $C$, that is  a non-trivial graph, its $N$
        children, $C_1, \ldots, C_N$, are all of the rigid vertex-
        maximal proper subgraphs of $C$. This makes Rule 2 implicit,
        and in fact, $C$ contains all the edges in the union of its
        children as well. The complete DR-plan is unique.
    A DR-plan is \dfn{optimal} if it minimizes the maximum fan-in over all nodes in the tree. It is not necessarily unique.
\end{definition}

% \begin{definition}
%     The \textbf{complete DR-plan} of graph $G$, $CompleteDRP(G)$, is the unique DR-plan but with a modified rule number 2: the children are \emph{all} of the trivial or well-constrained vertex-maximal proper subgraphs of that node $C$. By this, rule 3 is implicit for a complete DR-plan.
% \end{definition}

% \begin{definition}
%     The \textbf{optimal DR-plan} of graph $G$, $OptimalDRP(G)$, is the DR-plan that minimizes the maximum fan-in over all nodes in the tree. It is not necessarily unique.
% \end{definition}




%\emph{well-constrained} bar and joint graph with values $(V,E,w_V,w_E)$.
% \todo{I don't think the following is actually necessary to mention...} To keep problems meaningful we assume that $G$ is connected, as the methods discussed are used to solve systems of equations to establish linkages. It is not useful to solve for two bodies simultaneously if there are no constraints between them.
% THIS MEANS NO EDGES MAY BE LEFT OUT, ALSO.

\noindent
\note In Section \ref{sec:DRP} $G$ is assumed to be isostatic.
$Idc(G,X)$ is the subgraph of $G$ induced by the vertex set
$X\subseteq V$. For a subgraph $H=(W,F)$ of $G$, $Idc(G,H)=Idc(G,W)$.
In general, $C$ is the graph in an arbitrary node in $CompleteDRP(G)$.
$C_i$ is the $i^{\text{th}}$ child of $C$. It is implied that $C_i$ is
an isostatic vertex-maximal proper subgraph.

\subsection{Importance of an Optimal DR-plan}
The use of a DR-plan in designing especially self-similar qusecs is
immediately clear, as shown in Figures \ref{XX} and \ref{XX}
\todo{Reference?}. In fact, knowing an optimal DR-plan is crucial for
design and analysis of even non-self-similar qusecs and their
realizations with various desirable properties.

First, note that the basic task of \header{realizing given qusecs}
would require finding real solutions a large multivariate polynomial
system (of inequalities and equalities representing the constraints).
Without DR-planning, this requires double exponential time in the
number of variables (even if orientation type is specified). With DR-
planning, the complexity is dominated by the size of the largest
subsystem that is solved, or recombined from the solutions of its
child subsystems i.e.,the maximum fan-in occuring in a DR-plan. Thus
finding an optimal DR-plan is important for realization.

Next, using the DR-plan we obtain a \header{recursive block
decomposition of the rigidity matrix} with each block being the
rigidity matrix of an isostatic subgraph. The blocks intersect on
isostatic or trival subgraphs. This recursive block decomposition of
the rigidity matrix $R$ correspondingly \header{recursively decomposes
the stress vectors} $s$ that balance a given external stress $t$ since
$sR = t$, and moreover, the external stress vector $t$ now
\header{recursively ``distributes'' as external stresses} for each
isostatic subgraph of the DR-plan. When the rigidity matrix and stress
vectors are indeterminates, it can be used to design  qusecs
underlying graph and parameters $\delta$ to obtain an optimal
distribution of stresses on the geometric primitives. When the
realization of the qusec is given, this can be used to analyze the
distribution of a given external load.




%\subsection{DR-plan}
%%\subsection{Definitions}
%%\label{sec:prelim_defs}
%%
%%% TODO
%%
%%% A paragraph of preliminaries such as:
%%%     - rigidity matrix R,
%%%     - stress vectors S one coordinate per edge (w/ SR =E (external stress vector, 2 coordinates  per vertex));
%%%     - flex vectors F (2 coordinates per vertex) (right nullspace/kernel of R)
%%% Should precede the definition of
%%%     + independent,
%%%     + overconstrained (not independent),
%%%     - rigid (contains an independent set with sufficiently many eedges/constraints)
%%%     - isostatic (wellconstrained/minimally rigid), ,
%%%     - flexible (not rigid),
%%%     + underconstrained (independent and not rigid).
%%% After that, define DR plans etc.
%%
%%In Sections \ref{sec:DRP} and \ref{sec:recomb} a \textbf{graph} is a bar and joint system. $G=(V,E,w_V,w_E)$ is a set of vertices, $V$, and a set of edges, $E$, defined as a tuple of vertices from $V$ (undirected). Additionally, there are two weight functions, $w_V: V \to \mathbb{R}^+$ and $w_E: E \to \mathbb{R}^+$. The \textbf{density} of graph $G$ is $d(G) = \sum_{e\in E}{w_E(e)} - \sum_{v\in V}{w_V(v)}$.
%%
%%Given a constant $k$, a graph $G$ is:
%%\begin{itemize}
%%    \item \textbf{independent} (or \textbf{sparse}) if, for all non-trivial subgraphs $S\subseteq G$, $d(S) \leq k$.
%%
%%    \item \textbf{over-constrained} if it is not independent.
%%
%%    \item \textbf{well-constrained} (or \textbf{tight}) if $G$ is independent and $d(G)=k$.
%%
%%    \item \textbf{rigid} if there exists some spanning subgraph $S\subseteq G$ such that $S$ is well-constrained.
%%
%%    \item \textbf{under-constrained} if $G$ is independent and not rigid.
%%
%%    % \item \textbf{trivial} if (1) it is over-constrained graph and (2) all of its subgraphs are also trivial. Trivial graphs are input.
%%\end{itemize}
%%
%\textbf{Trivial} graphs are ill-defined in the general case. The only strict requirements are: (1) it must over-constrained and (2) all of its subgraphs are also trivial.
%
%% \begin{definition}
%%     A graph is \textbf{over-constrained} if, given constant $k$, there exists some
%%     % induced
%%     subgraph $S\subseteq G$ such that $d(S) > k$.
%% \end{definition}
%
%% \begin{definition}
%%     A graph is \textbf{well-constrained} if, given constant $k$, $d(G) = k$ and for all
%%     % induced
%%     subgraphs $S\subseteq G$, (1) $d(S)\leq k$, or (2) given a set of trivial graphs $T$, $S$ is isomorphic to one of the graphs in $T$ (i.e.\ $S$ is trivial).
%% \end{definition}
%
%
%
%% \begin{definition}
%%     A \textbf{trivial} graph is ill-defined in the general case. The only strict requirements are: (1) it must be an over-constrained graph and (2) all of its subgraphs are also trivial.
%%     \todo{Maybe leave the following out until it's needed later?}
%%     In the familiar geometric cases of $d$-dimensional space, all vertex weights are $d$, all edge weights are $1$, and constant $k= -{{d+1}\choose{2}}$. Trivial graphs for 2D would be a single vertex. Trivial graphs in 3D would be a vertex and an edge (2 vertices with an edge between). Etc. These trivial graphs capture the notion of the rotational symmetry that exists in geometric spaces.
%% \end{definition}
%
%
%\begin{definition}\label{def:drp}
%    The \textbf{decomposition-recombination \mbox{(DR-)} plan} of graph $G$, $DRP(G)$, is defined as a tree that has the following properties:
%    \begin{enumerate}
%        \item The root of the tree `contains' $G$.
%        \item For all nodes, $C$, that `contain' non-trivial graphs, its $N$ children, $C_1, \ldots, C_N$, are trivial or well-constrained vertex-maximal proper subgraphs of that node $C$.
%        \item The vertex set of $\bigcup_{i=1}^N{C_i}$ is the vertex set of $C$.
%        \item A node with a trivial subgraph is a leaf.
%    \end{enumerate}
%
%    % It can also be described recursively as: the root is $G$, its children are the trivial subgraphs and the DR-plans of its well-constrained vertex-maximal proper subgraphs whose union is $G$ itself.
%
%    An equivalent DAG can be constructed from this tree such that all nodes containing the same subgraphs are combined into one, with all edges preserved.
%
%    Note that nodes will be referred to interchangeably as ``the node that contains the (sub)graph $C$'' and as simply ``$C$''.
%
%    A DR-plan is \textbf{complete} if it satisfies the modified rule number 2: the children are \emph{all} of the trivial or well-constrained vertex-maximal proper subgraphs of that node $C$. This makes rule 3 implicit. The complete DR-plan is unique.
%
%    A DR-plan is \textbf{optimal} if it minimizes the maximum fan-in over all nodes in the tree. It is not necessarily unique.
%\end{definition}
%
%% \begin{definition}
%%     The \textbf{complete DR-plan} of graph $G$, $CompleteDRP(G)$, is the unique DR-plan but with a modified rule number 2: the children are \emph{all} of the trivial or well-constrained vertex-maximal proper subgraphs of that node $C$. By this, rule 3 is implicit for a complete DR-plan.
%% \end{definition}
%
%% \begin{definition}
%%     The \textbf{optimal DR-plan} of graph $G$, $OptimalDRP(G)$, is the DR-plan that minimizes the maximum fan-in over all nodes in the tree. It is not necessarily unique.
%% \end{definition}
%
%
%\subsection{Notation}
%
%In Sections \ref{sec:DRP} and \ref{sec:recomb}, $G$ is taken to be a \emph{well-constrained} bar and joint graph with values $(V,E,w_V,w_E)$.
%% \todo{I don't think the following is actually necessary to mention...} To keep problems meaningful we assume that $G$ is connected, as the methods discussed are used to solve systems of equations to establish linkages. It is not useful to solve for two bodies simultaneously if there are no constraints between them.
%% THIS MEANS NO EDGES MAY BE LEFT OUT, ALSO.
%
%$Idc(G,X)$ is the graph induced on $G$ with the vertex set $X\subseteq V$. This can also be overloaded such that, using graph $H=(W,F)$, $Idc(G,H)=Idc(G,W)$.
%
%$C$ is the graph in an arbitrary node in $CompleteDRP(G)$. $C_i$ is the $i^{\text{th}}$ child of $C$. By definition of a DR-plan, it is implied that $C_i$ is a well-constrained vertex-maximal proper subgraph.
