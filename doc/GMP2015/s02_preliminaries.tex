\section{Preliminaries}
\label{sec:prelim}

\subsection{Basics}
%TODO!

%Standard except for generality and ordering and the fact that I want to
%avoid the use of the word framework if possible. Do everything
%with constraint systems, linearization and genericity using indeterminates. 

%--general  geometric constraint system --

A $d$-dimensional {\it geometric constraint system} in general consists of

- a underlying {\it constraint graph} $G=(V,E)$, or more generally, a hypergraph, whose  vertices represents geometric primitives in $d$-dimension and whose (hyper)edges specifies the combinatorics of constraints between pairs/tuples of the geometric primitives.

- an {\it algebraic system of  constraint functions} from $E$ to the domain of constraint values


In this paper we study qusecs, a class of 2D geometric constrain system that is independent (defined below).


A familiar example of geometric constraint system is {\it linkage (barjoint system)},
where the underlying constraint graph is a simple graph and
the constraints are Euclidean distances on the edges of the graph.


%--Give example as
%linkage(barjoint system) 
\todo{-- refer to others (citations
+ pinned line incidence and multi-body-pin (in application
section)}

%--underlying constraint graph or hypergraph (common treatment)
\todo{???The (hyper)graph of a is sometimes refered to???}

%--linearization: Jacobian of constraint system,
%rigidity matrix of indeterminates

The algebraic system can usually be linearized at {\it generic} points
by taking the Jacobian of the system,
giving  the $d$-dimensional {\it rigidity matrix } $R_d(G)$ of a the (hyper)graph $G = (V,E)$, 
whose rows correspond to edges, columns correspond to coordinate positions of vertices,  entries are \emph{indeterminates} representing the coordinate position $\mathbf{p}(v) \in \mathbb{R}^d$ of the geometric primitives corresponding to vertices $v \in V$. 

%-- give example for the case of
%linkages

In the case of linkage, 
let $p_1(v), p_2(v), \ldots, p_d(v)$ represent the coordinate position $\mathbf{p}(v) \in \mathbb{R}^d$ of the joint corresponding to a vertex $v \in V$. 
The matrix $R_d(G)$ has one row for each edge $e \in E$ and $d$ columns for each vertex $v \in V$.  The row corresponding to $e = \{u,v\} \in E$ represents the bar connecting $p(u)$ and $p(v)$ and has $d$ non-zero indeterminate entries $\mathbf{p}(u)-\mathbf{p}(v)$ (resp.\ $\mathbf{p}(v)-\mathbf{p}(u)$) in the $d$ columns corresponding to $u$ (resp.  $v$) and zero in the other entries.  

%%Let $p_1(v), p_2(v), \ldots, p_d(v)$ represent the coordinate position $\mathbf{p}(v) \in \mathbb{R}^d$ of the joint corresponding to a vertex $v \in V$. 
%The matrix $R_d(G)$ has one row for each edge $e \in E$ and $d$ columns for each vertex $v \in V$.  The row corresponding to the hyperedge $e = \{v_1, v_2, \ldots, v_k\} \in E$ represents the constraint between the vertices $ \{v_1, v_2, \ldots, v_k\}$, 
%has $d$ non-zero indeterminate entries %$\mathbf{p}(u)-\mathbf{p}(v)$ (resp.\ $\mathbf{p}(v)-\mathbf{p}(u)$) 
%in the $d$ columns corresponding to each vertex $v_i \in e$ and zero in the other entries.  




%generic independence, infinitesimal rigidity
%(based on indeterminates)




A subset of edges $E'$, or a subgraph $(V', E')$, of a graph $G = (V, E)$ is said to be {\it independent} in $d$-dimensions, when the set of rows of $R_d(G)$ corresponding to $E'$ is generically independent, or independent for a generic instantiation of the indeterminate entries. 
This yields the $d$-dimensional generic {\it rigidity matroid} associated with a graph $G$. 

The graph is  {\it infinitesimally rigid} if %the rigidity matrix has full rank, that is, 
if the number of generically independent rows or the rank of $R_d(G)$ is maximal.
In the case of linkage, the maximal rank is $d|V| - {d+1 \choose 2}$, where ${d+1 \choose 2}$ is the number of rotational and translational degrees-of-freedom of a rigid body in $\mathbb{R}^d$ [?].
A vector in the kernel of the rigidity matrix is called an {\it infinitesimal motion vector}.

%--stress vectors and flex vectors
%(also based on indeterminates)


The left nullspace of the rigidity matrix is called the {\it space of stress}, consisting of the {\it flex vectors}. 


%--A generalized version of Asimow Roth (implicit fn thm)
%generic rigidity equivalent to infinitesimal rigidity (easy
%when the latter def is based on indeterminates)

 A generalized version of Asimow Roth (implicit fn thm) states that
 generic rigidity, i.e.\ there exist at most finitely many solutions to the system at generic points,  %to the algebraic system 
  is equivalent to  our definition  of infinitesimal rigidity based on indeterminates.



%--generic isostaticity (minimal rigidity, wellconstrainedness),
%flexibility (not rigid), underconstrained (independent
%and not rigid), overconstrained (dependent -- only briefly
%mentioned
%in this paper; all results are about independent - underconstrained or
%isostatic)

A graph $G = (V,E)$ is:
\begin{itemize}
	\item generically {\it isostatic (minimally rigid, well-constrained)} if $R_d(G)$ has full rank
	\item \textbf{rigid} if there exists some spanning subgraph $S\subseteq G$ such that $S$ is well-constrained.
	\item {\it flexible} if $G$ is not rigid
    \item \textbf{under-constrained} if $G$ is independent and not rigid.
    \item \textbf{over-constrained} if it is not independent.
\end{itemize}
In this paper all results are about independent - underconstrained or
isostatic


\todo{--rigidity matroid\\
--Combinatorial characterization of rigidity matroid
Example: Laman's theorem for bar-joint (refer to others)
Lovasz Yemini
Tay white and whiteley
(citations)
+ pinned line incidence and multi-body-pin (in application
section)}



% % % % % % % % % % % % % % % % % % % % % % % % % % % % %

%The problem is equivalent to combinatorially determining the generic rank of the 3-dimensional bar-joint rigidity matrix of a graph.  The $d$-dimensional bar-joint rigidity matrix of a graph $G = (V,E)$, denoted $R_d(G)$, is a matrix of indeterminates. Let $p_1(v), p_2(v), \ldots, p_d(v)$  represent the coordinate position $\mathbf{p}(v) \in \mathbb{R}^d$of the joint corresponding to a vertex $v \in V$.  . The matrix $R_d(G)$ has one row for each edge $e \in E$ and $d$ columns for each vertex $v \in V$.  The row corresponding to $e = \{u,v\} \in E$ represents the bar connecting $p(u)$ and $p(v)$ and has $d$ non-zero indeterminate entries $\mathbf{p}(u)-\mathbf{p}(v)$ (resp.\ $\mathbf{p}(v)-\mathbf{p}(u)$) in the $d$ columns corresponding to $u$ (resp.  $v$) and zero in the other entries.  
%
%A subset of edges $E'$, or a subgraph $(V', E')$, of a graph $G = (V, E)$ is said to be independent (we drop ``bar-joint'' from now on) in $d$-dimensions, when the set of rows of $R_d(G)$ corresponding to $E'$ is generically independent, or independent for a generic instantiation of the indeterminate entries. This yields the $d$-dimensional generic rigidity matroid associated with a graph $G$. The graph is rigid if the number of generically independent rows or the rank of $R_d(G)$ is maximal, i.e., $d|V| - {d+1 \choose 2}$, where ${d+1 \choose 2}$ is the number of rotational and translational degrees-of-freedom of a rigid body in $\mathbb{R}^d$ [?].




\subsection{DR-plan}
%\subsection{Definitions}
%\label{sec:prelim_defs}
%
%% TODO
%
%% A paragraph of preliminaries such as:
%%     - rigidity matrix R,
%%     - stress vectors S one coordinate per edge (w/ SR =E (external stress vector, 2 coordinates  per vertex));
%%     - flex vectors F (2 coordinates per vertex) (right nullspace/kernel of R)
%% Should precede the definition of
%%     + independent,
%%     + overconstrained (not independent),
%%     - rigid (contains an independent set with sufficiently many eedges/constraints)
%%     - isostatic (wellconstrained/minimally rigid), ,
%%     - flexible (not rigid),
%%     + underconstrained (independent and not rigid).
%% After that, define DR plans etc.
%
%In Sections \ref{sec:DRP} and \ref{sec:recomb} a \textbf{graph} is a bar and joint system. $G=(V,E,w_V,w_E)$ is a set of vertices, $V$, and a set of edges, $E$, defined as a tuple of vertices from $V$ (undirected). Additionally, there are two weight functions, $w_V: V \to \mathbb{R}^+$ and $w_E: E \to \mathbb{R}^+$. The \textbf{density} of graph $G$ is $d(G) = \sum_{e\in E}{w_E(e)} - \sum_{v\in V}{w_V(v)}$.
%
%Given a constant $k$, a graph $G$ is:
%\begin{itemize}
%    \item \textbf{independent} (or \textbf{sparse}) if, for all non-trivial subgraphs $S\subseteq G$, $d(S) \leq k$.
%
%    \item \textbf{over-constrained} if it is not independent.
%
%    \item \textbf{well-constrained} (or \textbf{tight}) if $G$ is independent and $d(G)=k$.
%
%    \item \textbf{rigid} if there exists some spanning subgraph $S\subseteq G$ such that $S$ is well-constrained.
%
%    \item \textbf{under-constrained} if $G$ is independent and not rigid.
%
%    % \item \textbf{trivial} if (1) it is over-constrained graph and (2) all of its subgraphs are also trivial. Trivial graphs are input.
%\end{itemize}
%
\textbf{Trivial} graphs are ill-defined in the general case. The only strict requirements are: (1) it must over-constrained and (2) all of its subgraphs are also trivial.

% \begin{definition}
%     A graph is \textbf{over-constrained} if, given constant $k$, there exists some
%     % induced
%     subgraph $S\subseteq G$ such that $d(S) > k$.
% \end{definition}

% \begin{definition}
%     A graph is \textbf{well-constrained} if, given constant $k$, $d(G) = k$ and for all
%     % induced
%     subgraphs $S\subseteq G$, (1) $d(S)\leq k$, or (2) given a set of trivial graphs $T$, $S$ is isomorphic to one of the graphs in $T$ (i.e.\ $S$ is trivial).
% \end{definition}



% \begin{definition}
%     A \textbf{trivial} graph is ill-defined in the general case. The only strict requirements are: (1) it must be an over-constrained graph and (2) all of its subgraphs are also trivial.
%     \todo{Maybe leave the following out until it's needed later?}
%     In the familiar geometric cases of $d$-dimensional space, all vertex weights are $d$, all edge weights are $1$, and constant $k= -{{d+1}\choose{2}}$. Trivial graphs for 2D would be a single vertex. Trivial graphs in 3D would be a vertex and an edge (2 vertices with an edge between). Etc. These trivial graphs capture the notion of the rotational symmetry that exists in geometric spaces.
% \end{definition}


\begin{definition}\label{def:drp}
    The \textbf{decomposition-recombination \mbox{(DR-)} plan} of graph $G$, $DRP(G)$, is defined as a tree that has the following properties:
    \begin{enumerate}
        \item The root of the tree `contains' $G$.
        \item For all nodes, $C$, that `contain' non-trivial graphs, its $N$ children, $C_1, \ldots, C_N$, are trivial or well-constrained vertex-maximal proper subgraphs of that node $C$.
        \item The vertex set of $\bigcup_{i=1}^N{C_i}$ is the vertex set of $C$.
        \item A node with a trivial subgraph is a leaf.
    \end{enumerate}

    % It can also be described recursively as: the root is $G$, its children are the trivial subgraphs and the DR-plans of its well-constrained vertex-maximal proper subgraphs whose union is $G$ itself.

    An equivalent DAG can be constructed from this tree such that all nodes containing the same subgraphs are combined into one, with all edges preserved.

    Note that nodes will be referred to interchangeably as ``the node that contains the (sub)graph $C$'' and as simply ``$C$''.

    A DR-plan is \textbf{complete} if it satisfies the modified rule number 2: the children are \emph{all} of the trivial or well-constrained vertex-maximal proper subgraphs of that node $C$. This makes rule 3 implicit. The complete DR-plan is unique.

    A DR-plan is \textbf{optimal} if it minimizes the maximum fan-in over all nodes in the tree. It is not necessarily unique.
\end{definition}

% \begin{definition}
%     The \textbf{complete DR-plan} of graph $G$, $CompleteDRP(G)$, is the unique DR-plan but with a modified rule number 2: the children are \emph{all} of the trivial or well-constrained vertex-maximal proper subgraphs of that node $C$. By this, rule 3 is implicit for a complete DR-plan.
% \end{definition}

% \begin{definition}
%     The \textbf{optimal DR-plan} of graph $G$, $OptimalDRP(G)$, is the DR-plan that minimizes the maximum fan-in over all nodes in the tree. It is not necessarily unique.
% \end{definition}


\subsection{Notation}

In Sections \ref{sec:DRP} and \ref{sec:recomb}, $G$ is taken to be a \emph{well-constrained} bar and joint graph with values $(V,E,w_V,w_E)$.
% \todo{I don't think the following is actually necessary to mention...} To keep problems meaningful we assume that $G$ is connected, as the methods discussed are used to solve systems of equations to establish linkages. It is not useful to solve for two bodies simultaneously if there are no constraints between them.
% THIS MEANS NO EDGES MAY BE LEFT OUT, ALSO.

$Idc(G,X)$ is the graph induced on $G$ with the vertex set $X\subseteq V$. This can also be overloaded such that, using graph $H=(W,F)$, $Idc(G,H)=Idc(G,W)$.

$C$ is the graph in an arbitrary node in $CompleteDRP(G)$. $C_i$ is the $i^{\text{th}}$ child of $C$. By definition of a DR-plan, it is implied that $C_i$ is a well-constrained vertex-maximal proper subgraph.
