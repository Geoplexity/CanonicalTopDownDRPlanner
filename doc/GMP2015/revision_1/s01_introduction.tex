\section{Response to reviewers}

\cutout{This means} the text was removed.
\addin{This means} the text was added.
\notetoreviewer{This means} it is a note to the reviewer, and would not appear in real document.
% \movedfrom{This means} the text was moved from another part of the document.

\subsection{Reviewer 1}
No suggestions made.

\subsection{Reviewer 2}
% \noindent
% Changes made:
\begin{itemize}
  \item We added a new example in figures to illustrate definitions/proofs/comparisons. And we added a new conjecture relating this method to the Modified Frontier Algorithm, the most recent and best current algorithm.
  \item Abstract was shortened and clarified.
  \item Basic definitions from Section \ref{sec:prelim} were moved to the new \ref{sec:appendix:defs}. Definition of qusecs was made more prominent, left in Section \ref{sec:prelim}. Slight modification to definition of DR-plan.
  \item Subsections \ref{sec:cont} and \ref{sec:intro:prevwork} were flipped. This gives readers an idea of the contributions before they read about the current state of the field.
  \item Section \ref{sec:DRP} had some proofs moved back in from the appendix. Section \ref{sec:recomb} had a proof moved back in from the appendix. Section \ref{sec:conclusion} had future work moved back in from the appendix.
  \item Numerous grammatical, syntactical, and spelling errors were fixed. Made hyphen usage consistent.
  \item Images were properly aligned and cropped.
\end{itemize}

\subsection{Reviewer 3}
We emailed the co-chairs, and they replied that it would fit in the conference, leaving it at our discretion to withdraw. We have decided to continue.




\section{Introduction}
\label{sec:intro}

\newcommand{\seedefs}{(formally defined in \ref{sec:appendix:defs})}
\newcommand{\seedefsb}{(see \ref{sec:appendix:defs} for definitions)}
\newcommand{\seedefsc}{See \ref{sec:appendix:defs} for formal definitions}
\newcommand{\seedefsd}{see \ref{sec:appendix:defs} for definitions}
\newcommand{\seedefsprelim}{(formally defined in Section \ref{sec:prelim})}


Some properties of natural and engineered materials can be analyzed by treating them as two dimensional (2D) layers. As illustrated by the examples below, the structure within each layer is often self-similar~\cite{2012arXiv1204.6389G} spanning multiple scales; generally aperiodic and quasi-uniform within any one scale; and consists of a few repeated motifs appearing in disordered arrangements. Each layer is not necessarily planar, i.e., it consists of multiple, inter-constraining planar (genus 0) monolayers. Furthermore, a layer is often  either
\vemph{isostatic or underconstrained}\cutout{,} \addin{(}\vemph{not self-stressed}\addin{,}
\cutout{(}\seedefsd).
\cutout{These properties are consistent with a self-assembled 2D structure that minimizes mass, optimally distributes external stresses and itself participates in the assembly of diverse and multifunctional, larger 2D structures.}
\addin{A variety of quasi-uniform (aperiodic) and self-similar, layered material structures are a natural consequence of design objectives such as rigidity or flexibility, minimizing mass, optimally distributing external stresses and participating in the assembly of diverse and multifunctional, larger  structures.}

\noindent
\textbf{Note on Scope:} In this manuscript we only study finite 2D structures. \dfn{Self-similarity} refers to the result of finitely many levels of hierarchy or subdivision in an iterated scheme to generate self-similar structures.



% \FigInit{}{fig:material_examples}
% \FigThreeSubfig%
%   {img/Chlamydomonas_TEM_17}
%   {Cross section of the Chlamydomonas algae axoneme, a cilia composed of microtubules~\cite{wikimediacommons2007cilia}.}
%   {fig:material_examples:microtubule}%
%   %
%   {img/ligten2}
%   {Cross section of a tendon displaying the hierarchical structure~\cite{lecture_biosolid_mechanics}.}
%   {fig:material_examples:tendon}%
%   %
%   {img/Rothemund-DNA-SierpinskiGasket}
%   {A DNA array exhibiting the Sierpinski triangle~\cite{wikimediacommons2007dna}.}
%   {fig:material_examples:sierpinski}

\ClearMyMinHeight
\SetMyMinHeight{.32}{img/Chlamydomonas_TEM_17}
\SetMyMinHeight{.32}{img/ligten2}
\SetMyMinHeight{.32}{img/Rothemund-DNA-SierpinskiGasket}

\begin{figure*}\centering%
  %
  \begin{subfigure}{0.32\linewidth}\centering
    \includegraphics[height=\myMinHeight]{img/Chlamydomonas_TEM_17}
    \caption{}\label{fig:material_examples:microtubule}
  \end{subfigure}%
  %
  \hfill
  \begin{subfigure}{0.32\linewidth}\centering
    \includegraphics[height=\myMinHeight]{img/ligten2}
    \caption{}\label{fig:material_examples:tendon}
  \end{subfigure}%
  %
  \hfill
  \begin{subfigure}{0.32\linewidth}\centering
    \includegraphics[height=\myMinHeight]{img/Rothemund-DNA-SierpinskiGasket}
    \caption{}\label{fig:material_examples:sierpinski}
  \end{subfigure}%
  %
  \caption{(\ref{fig:material_examples:microtubule}) Cross section of the Chlamydomonas algae axoneme, a cilia composed of microtubules~\cite{wikimediacommons2007cilia}. (\ref{fig:material_examples:tendon}) Cross section of a tendon displaying the hierarchical structure~\cite{lecture_biosolid_mechanics}. (\ref{fig:material_examples:sierpinski}) A DNA array exhibiting the Sierpinski triangle~\cite{wikimediacommons2007dna}.}\label{fig:material_examples}
\end{figure*}%


% \FigAddSubfig%
%   {0.3}
%   {img/Chlamydomonas_TEM_17}
%   {Cross section of the Chlamydomonas algae axoneme, a cilia composed of microtubules~\cite{wikimediacommons2007cilia}.}
%   {fig:material_examples:microtubule}%
% \FigAddSubfig%
%   {0.3}
%   {img/ligten2}
%   {Cross section of a tendon displaying the hierarchical structure~\cite{lecture_biosolid_mechanics}.}
%   {fig:material_examples:tendon}%
% \FigAddSubfig%
%   {0.3}
%   {img/Rothemund-DNA-SierpinskiGasket}
%   {A DNA array exhibiting the Sierpinski triangle~\cite{wikimediacommons2007dna}.}
%   {fig:material_examples:sierpinski}
% \FigDisplay{}{fig:material_examples}


Examples of such materials (See Figure \ref{fig:material_examples})
include:
%
\begin{enumerate}
    \item \label{materialexample1} Cross-sections of microtubule structures~\cite{microtubule_necklace} (Figure \ref{fig:material_examples:microtubule}) e.g., in ciliary membranes and transitions~\cite{microtubule_cilia}.

    \item \label{materialexample2} Cross-sections of organic tissue with hierarchical structure, e.g. compact bone and tendon (Figure \ref{fig:material_examples:tendon}).

    \item \label{materialexample3} Crosslinked cellulose or collagen microfibril monolayers e.g., in cell-walls~\cite{wikimediacommons2010afm}~\cite{wikimediacommons2007plant}, as well as crosslinked actin filaments in the cytoskeleton matrix. See Section \ref{sec:pinnedline}.

    \item \label{materialexample4} More recent, engineered examples, including disordered graphene layers~\cite{Graphene1}~\cite{Graphene2} sometimes reinforced by microfibrils; and DNA assemblies including a recent Sierpinski gasket~\cite{self_assembly_sierpinski}, bringing other self-similar structures~\cite{wikimediacommons2012subdivision} within reach.

    \item \label{materialexample5} Silica bi-layers~\cite{silica_bilayers}, glass~\cite{sructure_of_2d_glass}, and materials that behave like assemblies of 2D particles under non-overlap constraints, i.e, like jammed disks on the plane~\cite{jammed_disks}. See Section \ref{sec:bodypin}.
\end{enumerate}
%
In order to study structural and mechanical properties of a material layer, it is natural to model a material layer as a solution or realization of a geometric constraint system of appropriate types of geometric primitives, under metric or algebraic constraints. Such 2D \dfn{qusecs}, quasi-uniform or self-similar constraint systems \seedefsprelim, can be used to understand or design material layers (their solutions) with desired properties.





\subsection{Contributions and Organization}
\label{sec:cont}

\notetoreviewer{The position of this section and following section were swapped.}

The contributions of this paper are the following.
\begin{itemize}
  \item \addin{In Section \ref{sec:prelim}, we give definitions and give the new notion of qusecs and the canonical/optimal DR-plan. This is relevant to Examples \ref{materialexample1} and \ref{materialexample2}, where we model them as bar-joint systems and discuss achieving isostaticity, distribution of stresses in self-similar, and other important concepts.}

  \item In Section \ref{sec:DRP}, we navigate the NP-hardness barrier (discussed in the following subsection), for finding optimal DR-plans by defining a so-called \dfn{canonical} DR-plan and showing a strong Church-Rosser property: \vemph{all canonical DR-plans for isostatic or underconstrained 2D qusecs are optimal}.

  \item Also in Section \ref{sec:DRP}, we give an efficient (\candrpcomplexity) algorithm to find a canonical (and hence optimal) DR-plan for all 3 types of 2D qusecs mentioned above (Sections \ref{sec:DRP}, \ref{sec:bodypin}, and \ref{sec:pinnedline}). The canonical DR-plan elucidates the essence of the NP-hardness of finding optimal DR-plans for over-constrained systems. Furthermore, our optimal/canonical DR-plan satisfies desirable properties such as the previously studied Cluster Minimality \cite{hoffman2001decompositionI} \addin{(see Figure~\ref{fig:demo_graph:clustmindrp})}.

  \item In Section \ref{sec:recomb}, we give a method to deal with the algebraic complexity of recombining the realizations or solutions of child subsystems into a solution of the parent system \cite{sitharam2010optimized,sitharam2006well,sitharam2010reconciling}. Specifically, we define the problem of minimally modifying the indecomposable recombination system so that it becomes decomposable via a small DR-plan and yet preserves the original solutions in an efficiently searchable manner. In Section \ref{sec:table}, we show formal connection to well known problems such as optimal completion of underconstrained systems \cite{joan-arinyo2003transforming,sitharam2005combinatorial,gao2006ctree} and to find paths within the connected components. When the modifications are bounded, we obtain new, efficient algorithms for realizing both isostatic and underconstrained qusecs by leveraging recent results about Cayley parameters in \cite{sitharam2010convex,sitharam2011cayleyI,sitharam2011cayleyII} (see Sections \ref{sec:2-tree-reduction} and \ref{sec:tdecomp}).

  \item In Section \ref{sec:bodypin} and \ref{sec:pinnedline}, we briefly describe applications of the above techniques to modeling, analyzing, and designing specific properties in 2D material layers~\cite{Jackson2008bodypin}.
  \cutout{For Examples 1 and 2 (achieving isostaticity, distribution of stresses in self-similar, bar-joint systems); Example 3 (canonical and optimal DR-plans for pinned line incidence systems [X]) and Examples 4 and 5 (boundary-conditions for achieving various desired properties of body-hyperpin systems).}
  \addin{For Examples \ref{materialexample4} and \ref{materialexample5}, we discuss boundary-conditions for achieving various desired properties of body-hyperpin systems. For Example \ref{materialexample3}, we discuss canonical and optimal DR-plans for pinned line incidence systems \cite{sitharam2014incidence}).}

  \item We intend to make software implementation and videos available upon request and publicly available for the final version of the paper.
\end{itemize}





\subsection{Previous Work on Relevant 2D Geometric Constraint Systems}
\label{sec:intro:prevwork}

We now briefly survey existing techniques for studying 2D qusecs,  many of which are \dfn{bar-joint} systems (Examples 1\cutout{,} \addin{and} 2 above, see Sections \ref{sec:prelim}, \ref{sec:DRP}, \addin{and} \ref{sec:recomb}), \dfn{body-hyperpin} systems (Example 4\cutout{,} \addin{and} 5, see Section \ref{sec:bodypin}) or \dfn{pinned-line incidence} systems (Example 3, see Section \ref{sec:pinnedline}). The limitations of these techniques directly motivate the contributions of this paper.

\medskip\noindent
\header{(i) Finding (vertex)-maximal generically rigid subsystems}
Fast, graph-based algorithms exist (pebble-game \cite{Jacobs:1997:PG,lee2005finding}), for locating all maximal, \dfn{generically rigid} subsystems \seedefs. When the input itself is rigid, these algorithms do nothing, i.e., compute the identity function.

However, both for self-similar or just aperiodic 2D qusecs, it is imperative to recursively decompose rigid systems into their rigid subsystems, down to the level of geometric primitives, in order to understand or design properties at all scales, such as \seedefs\ \dfn{rigidity}, \dfn{flexes}, distribution of \dfn{external stresses}, boundary conditions for \dfn{isostaticity}, as well as behavior under constraint variations.

\medskip\noindent
\header{(ii) Optimal Recursive Decomposition (DR-planning).}
Recursive decomposition of geometric constraint systems has been formalized \cite{hoffman2001decompositionI,hoffman2001decompositionII} and well-studied \cite{jermann2006decomposition,sitharam2005combinatorial} as the \dfn {Decomposition-Recombination (DR-) planning} problem \seedefsprelim. For the abovementioned classes of 2D qusecs, generic rigidity is a combinatorial property and hence each level of the decomposition should, in principle, be achievable by a graph-based algorithm as in (1), without involving geometric information in the constraint system. Since many such decompositions can exist for a given constraint system, criteria defining desirable or optimal DR-plans and DR-planning algorithms were given in \cite{hoffman2001decompositionI}. An \dfn{optimal DR-plan} is one that minimizes the \dfn{size} \seedefsprelim, i.e., the maximum number of child subsystems of any parent system. Being exponential \addin{in} the size, the complexity of solving the parent constraint system \cutout{from the solutions of the of the child systems, is overwhelmingly dominated by it} \addin{is overwhelmingly dominated by the complexity of solving the child systems}.

However, for general 2D qusecs that could be overconstrained, even when restricted to bar-joint systems, the optimal DR-planning problem was shown to be NP-hard \cite{lomonosov2004graph}.

\medskip\noindent
\header{(iii) DR-plans for special classes and with other
criteria.}
For a special class of 2D qusecs, namely \dfn{tree-decomposable} systems  \cite{fudos1997graph,owen1991algebraic,joan-arinyo2004revisiting}  common in computer aided mechanical design, (which includes ruler-and-compass and Henneberg-I constructible systems), all DR-plans turn out to be optimal. This satisfies the so-called \dfn {Church-Rosser} criterion, leading to highly efficient DR-planning algorithms. For general 2D qusecs, alternate criteria were suggested such as \dfn{cluster minimality} requiring parent systems to be composed of a minimal set of at least 2 rigid proper subsystems (i.e., no proper subset forms a rigid system); and \dfn{proper maximality}, requiring child subsystems to be maximal rigid proper subsystems of the parent system. \seedefsc.

While polynomial time algorithms were given to generate DR-plans meeting the cluster minimality criterion \cite{hoffman2001decompositionI}, no such algorithm is known for the latter criterion.


\medskip\noindent
\header{(iv) Optimal Recombination and Solution Space Navigation.}
For the entire DR-plan, finding all desired solutions is barely tractable even if recombination of solved subsystems is easy for each indecomposable parent system in the DR-plan. This is because even for the simplest, highly decomposable systems with small DR-plans, the problem of finding even a single solution to the input system at the root of the DR-plan is NP-hard  \cite{saxe1979embeddability} and there is a combinatorial explosion of solutions \cite{borcea2004number}. Typically, however, the desired solution has a given orientation type, in which case, the crux of the difficulty is concentrated in the algebraic complexity of (re)combining child system solutions\cutout{,} to give a solution to the parent system at any given level of the DR-plan. For fairly general 3D constraint systems, there are algorithms with formal guarantees that uncover underlying matroids to combinatorially obtain an optimal parameterization to minimize the algebraic complexity of the indecomposable parent (sub)systems that occur in the DR-plan \cite{sitharam2010optimized,sitharam2006well,sitharam2010reconciling}, provided the DR-plan meets some of the abovementioned criteria.

However, the generality of these algorithms trades-off against efficiency, and despite the optimization, the best algorithms can still take exponential time in the number of child subsystems (which can be arbitrarily large even for optimal DR-plans) in order to guarantee all solutions of a given orientation type, even for a single (sub)system in a DR-plan. They are prohibitively slow in practice. We note that, utilizing the DR-plan and optimal recombination as a principled basis, high performance heuristics and software exists \cite{sitharam2006solution} to tame combinatorial explosion via user intervention.


\medskip\noindent
\header{(v) Configuration Spaces of Underconstrained Systems.}
For underconstrained 2D bar-joint and body-hyperpin qusecs obtained from various subclasses of tree-decomposable systems, algorithms have been developed to complete them into isostatic systems \cite{joan-arinyo2003transforming,sitharam2005combinatorial,gao2006ctree,sitharam2010convex} and to find paths within the connected components \cite{sitharam2011cayleyI,hidalgo2011reachability} of standard Cartesian configuration spaces. Most of the algorithms with formal guarantees leverage Cayley configuration space theory \cite{sitharam2010convex,sitharam2011cayleyI,sitharam2011cayleyII} to characterize subclasses of graphs and additional constraints that control combinatorial explosion, and provide faithful bijective representation of connected components and paths. These algorithms have decreasing efficiency as the subclass of systems gets bigger, with highest efficiency for underlying partial 2-tree graphs (alternately called\cutout{,} tree-width 2, series-parallel, \addin{and} $K_4$ minor avoiding), moderate efficiency for 1 degree-of-freedom (dof) graphs with low Cayley complexity (which include common linkages such as the Strandbeest, Limacon and Cardioid), and decreased efficiency for general 1-dof tree-decomposable graphs. While software suites exist  \cite{keycurriculum1995geometer,porta2014open,siemens1999d,todd2007geometry}, no such formal algorithms and guarantees are known for non-tree-decomposable systems.

% \medskip\noindent
% \note All proofs appear in the Appendix.
% \medskip

\cutout{
\noindent
\textbf{Note:} All proofs appear in the Appendix.
% \sidenote{All proofs appear in X.}
}
