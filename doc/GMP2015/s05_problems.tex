\section{Problem Relationships and Unification}
\label{sec:table}
In this section we provide a unified view of  the various problems
studied in the previous 2 sections, along with formal reductions
between them, and relationships to other known problems and results as
well as open questions.
%
\subsection{Special classes of small DR-plans}
As seen in the previous section, 2-trees and tree-decomposable graphs
have not only small, but also special DR-plans that permit easy
solving --- essentially a single quadratic at a time.

The \dfn{restricted optimal DR-planning problem} requires DR-plans of
one of these types, which reduces to recognizing if the input graph is
a 2-tree or a tree-decomposable graph for which simple near-linear
time algorithms are available \cite{valdes1979recognition,fudos1997graph} and the DR-plan is a
by-product output of the recognition algorithm.

In the recombination setting, the corresponding \dfn{restricted
OMD$_k$  problem} requires the reduced graph $G'$ and its isostatic
completion $H$ to be 2-trees as in Section \ref{sec:2-tree-reduction}
or to be a low Cayley complexity tree-decomposable graph as in Section
\ref{sec:tdecomp}. Clearly these problems have deterministic
polynomial time algorithms, the algorithms run in time exponential in
$k$.

We discuss the complexity of the restricted
OMD problem (when $k$ is part of the input)
in Section \ref{sec:futurework}.

%In the recombination setting, the corresponding \dfn{restricted
%OMD$_k$  problem} requires the reduced graph $G'$ and its isostatic
%completion $H$ to be 2-trees as in Section \ref{sec:2-tree-reduction}
%or to be a low Cayley complexity tree-decomposable graph as in Section
%\ref{sec:tdecomp}. Clearly these problems have deterministic
%polynomial time algorithms, the algorithms run in time exponential in
%$k$. As a result, when $k$ is part of the input, the \dfn{restricted
%OMD problem} has the potential to be difficult. For example, when the
%isostatic completion $H$ is required to be a 2-tree the restricted OMD
%problem is reducible to the maximum spanning series-parallel subgraph
%problem shown by \cite{cai1993spanning} to be NP-complete even if the input
%graph is planar of maximum degree at most 6. However, since the OMD
%problem has other input restrictions such as not having any proper
%isostatic subgraphs, it is not clear if the reverse reduction exists
%and hence it is unclear whether the OMD problem is NP-complete. The
%same holds for the restricted OMD problem where the isostatic
%completion $H$ is required to be a tree-decomposable graph of low
%Cayley complexity (i.e, have special, small DR-plans). One potential
%obstacle to an indecomposable graph $G$'s membership in the restricted
%OMD$_k$ for small $k$ is if $G$ is tri-connected and has large girth.
%In fact, 6-connected (hence rigid) graphs with arbitrarily large girth
%have been constructed in \cite{servatius2000rigidity}.
%
\subsection{Optimal Modification, Completion and Recombination: formal
connections}
%
The OMD problem is closely related to a well-studied problem of
completion of an under-constrained system to an isostatic one with a
small DR-plan.
\begin{observation}\label{obs:OC_to_OMD}
    The (decision version of) the \textbf{optimal completion problem
    (OC)} from \cite{sitharam2005combinatorial,joan-arinyo2003transforming,zhang-gao2006well} is OMD$_0$.
\end{observation}
In fact, a \dfn{restricted OC} problem was studied by \cite{joan-arinyo2003transforming}
requiring the completion to be tree-decomposable.
%It is not clear if
%the reverse reduction holds, i.e, is OMD reducible to OC?

We now connect the OMD problem to the informal \dfn{optimal
recombination (OR)} problem mentioned as motivation at the beginning
of Section \ref{sec:recomb}.

In order to connect the OR problem to OMD, when the input graph is
the isostatic graph at the DR-plan root, we do not consider the case
where the two child \dfn{solved subgraphs} (corresponding to already
solved subsystems) have a nontrivial intersection (in this case the
recombination is trivial). We only consider the case where no two
child solved subgraphs (resp. two root subgraphs when the input graph
is under-constrained) share more than 1 vertex. We replace such solved
subgraphs  by isostatic graphs as follows. If a solved subgraph shares
at most one vertex with the remainder of the graph, simply replace it
by an edge one of whose endpoints  is the shared vertex. Otherwise,
replace it by  a 2-tree graph of the shared vertices. Finally, we add
the additional restriction to the OM Problem that when any edge in a
solved subgraph is chosen among the $k$ edges to be removed, in fact
the entire solved subgraph must be removed  and all of its edges must
be counted in $k$.

This reduction is used also for transferring  algorithms for optimal
DR-planning, recombination, completion, OMD, and other problems
from bar-joint systems to
so-called \dfn{body-hyperpin}, defined in Section \ref{sec:bodypin},
by showing that the problems for the latter are reduced
to the corresponding problems on bar-joint systems.

%\note This section deals only with bar-joint systems, however,
%extensions of several of these problems, approaches and solutions to
%other qusecs beyond  bar-joint - specifically pinned line incidence
%systems and multi-body-pin systems will be expanded in Sections
%\ref{sec:pinnedline} and \ref{sec:bodypin}.
