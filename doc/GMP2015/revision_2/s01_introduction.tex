\section{Introduction}
\label{sec:intro}

\newcommand{\seedefs}{(formally defined in \ref{sec:appendix:defs})}
\newcommand{\seedefsb}{(see \ref{sec:appendix:defs} for definitions)}
\newcommand{\seedefsc}{See \ref{sec:appendix:defs} for formal definitions}
\newcommand{\seedefsd}{see \ref{sec:appendix:defs} for definitions}
\newcommand{\seedefsprelim}{(formally defined in Section \ref{sec:prelim})}



\todo{Main overview/intro.}
Geometric constraint systems are ubiquitous in material and mechanical engineering, robotics, bioinformatics, and many other fields. However, the design and analysis of these systems is made difficult by i

\subsection{Recursive Decomposition Problem}
The first problem we consider is the optimal recursive decomposition problem of highly general geometric constraint systems with the sole requirement that local rigidity is characterized as a sparsity condition of an underlying (hyper)graph. We call these systems \nameit. This characterization has broad applicability in practice, and we provide several concrete examples.

Such decomposition is crucial for realizing and designing the geometric constraint system and its properties such as rigidity, stability, and stress response to external loads. In the literature, this problem is called the optimal decomposition recombination (DR-) planning problem, which, in general, is NP-hard. Therefore, previous algorithms for the problem did not consider optimality and instead focused on other desirable criteria of the output DR-plan. By concentrating on geometric constraint system that are not overconstrained, we give a polynomial time algorithm for optimal DR-planning by capturing the notion of a canonical DR-plan; we show that all canonical DR-plans are optimal for systems that are not overconstrained. This immediately gives a polynomial time algorithm for obtaining a DR-planning that is not only optimal but also has the previously studied desirable properties. In fact, we conjecture that one of the earlier algorithms also produces an optimal DR-plan and that every optimal DR-plan is in some formal sense canonical.


% Knowing an optimal DR-plan is crucial when working with isostatic geometric constraint systems with an underlying sparsity condition \nameit. It allows for efficient design and analysis of these systems with various desirable properties. The importance of an optimal plan is particularly evident when these systems \nameit are self-similar or quasi-uniform, shown in Figures \ref{fig:c2c3ofk33s} and \ref{fig:bodypindrp}, as the complexity of the plan is even further reduced.

When realizing such a system \nameit, the task would require finding real solutions to a large multivariate polynomial system (of inequalities and equalities representing the constraints). Without DR-planning, this requires double exponential time in the number of variables (even if orientation type is specified). With DR-planning, the complexity is dominated by the size of the largest subsystem that is solved, or recombined from the solutions of its child subsystems, i.e., the maximum fan-in occurring in a DR-plan. Thus, finding an optimal DR-plan is important for realization.


\subsection{Recombination Problem}
\todo{The second problem we consider...}
We additionally use the notion of Cayley complexity to quantify the difficulty of each step of recombination. Once a subsystem is found that has a convex Cayley configuration space, the space can be efficiently searched to find a realization that satisfies the additional constraints of the original system.
This allows for even greater reduction of the complexity, by realizing large, indecomposable systems in a way that avoids working with large systems of equations.



\subsection{Introducing Qusecs}
When working on these problems, there is a large, natural class of constraint systems that we call \dfn{qusecs}, a contraction of ``quasi-uniform or self-similar constraint systems''.
%
Some properties of natural and engineered materials can be analyzed by treating them as two dimensional (2D) layers. As illustrated by the examples below, the structure within each layer is often: self-similar\footnote{In this manuscript we only study finite 2D structures. \dfn{Self-similarity} refers to the result of finitely many levels of hierarchy or subdivision in an iterated scheme to generate self-similar structures.}~\cite{2012arXiv1204.6389G}, spanning multiple scales; generally aperiodic and quasi-uniform within any one scale; and composed of a few repeated motifs appearing in disordered arrangements.
Note that a layer is not necessarily planar, it can consist of multiple, inter-constraining planar (genus 0) monolayers. Furthermore, a layer is often  either \vemph{isostatic} or \vemph{underconstrained} (\vemph{not self-stressed}, \seedefsd). A variety of quasi-uniform (aperiodic) and self-similar, layered material structures are a natural consequence of design objectives such as rigidity or flexibility, minimizing mass, optimally distributing external stresses and participating in the assembly of diverse and multifunctional, larger  structures.
%
The importance of an optimal DR-plan is particularly evident when a system is a qusecs. The quasi-uniform or self-similar properties mean that solutions for one subsystem can solve other subsystems, thus causing further reduction in the complexity of the plan. This is shown in Figures \ref{fig:c2c3ofk33s} and \ref{fig:bodypindrp}.



% \FigInit{}{fig:material_examples}
% \FigThreeSubfig%
%   {img/Chlamydomonas_TEM_17}
%   {Cross section of the Chlamydomonas algae axoneme, a cilia composed of microtubules~\cite{wikimediacommons2007cilia}.}
%   {fig:material_examples:microtubule}%
%   %
%   {img/ligten2}
%   {Cross section of a tendon displaying the hierarchical structure~\cite{lecture_biosolid_mechanics}.}
%   {fig:material_examples:tendon}%
%   %
%   {img/Rothemund-DNA-SierpinskiGasket}
%   {A DNA array exhibiting the Sierpinski triangle~\cite{wikimediacommons2007dna}.}
%   {fig:material_examples:sierpinski}

\ClearMyMinHeight
\SetMyMinHeight{.32}{img/Chlamydomonas_TEM_17}
\SetMyMinHeight{.32}{img/ligten2}
\SetMyMinHeight{.32}{img/Rothemund-DNA-SierpinskiGasket}

\begin{figure*}\centering%
  %
  \begin{subfigure}{0.32\linewidth}\centering
    \includegraphics[height=\myMinHeight]{img/Chlamydomonas_TEM_17}
    \caption{}\label{fig:material_examples:microtubule}
  \end{subfigure}%
  %
  \hfill
  \begin{subfigure}{0.32\linewidth}\centering
    \includegraphics[height=\myMinHeight]{img/ligten2}
    \caption{}\label{fig:material_examples:tendon}
  \end{subfigure}%
  %
  \hfill
  \begin{subfigure}{0.32\linewidth}\centering
    \includegraphics[height=\myMinHeight]{img/Rothemund-DNA-SierpinskiGasket}
    \caption{}\label{fig:material_examples:sierpinski}
  \end{subfigure}%
  %
  \caption{(\ref{fig:material_examples:microtubule}) Cross section of the Chlamydomonas algae axoneme, a cilia composed of microtubules~\cite{wikimediacommons2007cilia}. (\ref{fig:material_examples:tendon}) Cross section of a tendon displaying the hierarchical structure~\cite{lecture_biosolid_mechanics}. (\ref{fig:material_examples:sierpinski}) A DNA array exhibiting the Sierpinski triangle~\cite{wikimediacommons2007dna}.}\label{fig:material_examples}
\end{figure*}%


% \FigAddSubfig%
%   {0.3}
%   {img/Chlamydomonas_TEM_17}
%   {Cross section of the Chlamydomonas algae axoneme, a cilia composed of microtubules~\cite{wikimediacommons2007cilia}.}
%   {fig:material_examples:microtubule}%
% \FigAddSubfig%
%   {0.3}
%   {img/ligten2}
%   {Cross section of a tendon displaying the hierarchical structure~\cite{lecture_biosolid_mechanics}.}
%   {fig:material_examples:tendon}%
% \FigAddSubfig%
%   {0.3}
%   {img/Rothemund-DNA-SierpinskiGasket}
%   {A DNA array exhibiting the Sierpinski triangle~\cite{wikimediacommons2007dna}.}
%   {fig:material_examples:sierpinski}
% \FigDisplay{}{fig:material_examples}


% Examples of such materials (See Figure \ref{fig:material_examples}) include:
% In order to study structural and mechanical properties of a material layer, it is natural to model a material layer as a solution or realization of a geometric constraint system of appropriate types of geometric primitives, under metric or algebraic constraints.
% Such 2D \dfn{qusecs}, quasi-uniform or self-similar constraint systems \seedefsprelim, can be used to understand or design material layers (their solutions) with desired properties.
Some materials that are readily modeled as qusecs include:
%
\begin{enumerate}
    \item \label{materialexample1} Cross-sections of microtubule structures~\cite{microtubule_necklace} (Figure \ref{fig:material_examples:microtubule}) e.g., in ciliary membranes and transitions~\cite{microtubule_cilia}.

    \item \label{materialexample2} Cross-sections of organic tissue with hierarchical structure, e.g. compact bone and tendon (Figure \ref{fig:material_examples:tendon}).

    \item \label{materialexample3} Crosslinked cellulose or collagen microfibril monolayers e.g., in cell-walls~\cite{wikimediacommons2010afm}~\cite{wikimediacommons2007plant}, as well as crosslinked actin filaments in the cytoskeleton matrix. See Section \ref{sec:pinnedline}.

    \item \label{materialexample4} More recent, engineered examples, including disordered graphene layers~\cite{Graphene1}~\cite{Graphene2} sometimes reinforced by microfibrils; and DNA assemblies including a recent Sierpinski gasket~\cite{self_assembly_sierpinski} (Figure \ref{fig:material_examples:sierpinski}), bringing other self-similar structures~\cite{wikimediacommons2012subdivision} within reach.

    \item \label{materialexample5} Silica bi-layers~\cite{silica_bilayers}, glass~\cite{sructure_of_2d_glass}, and materials that behave like assemblies of 2D particles under non-overlap constraints, i.e, like jammed disks on the plane~\cite{jammed_disks}. See Section \ref{sec:bodypin}.
\end{enumerate}



% \subsection{old}





\subsection{Organization and Contributions}
\label{sec:cont}

\todo{In Section \ref{sec:prelim}, we give definitions and give the new notion of qusecs and the canonical/optimal DR-plan. This is relevant to Examples \ref{materialexample1} and \ref{materialexample2}, where we model them as bar-joint systems and discuss achieving isostaticity, distribution of stresses in self-similar, and other important concepts.}

% In Section \ref{sec:DRP}, we navigate the NP-hardness barrier (discussed in the following subsection), for finding optimal DR-plans by defining a so-called \dfn{canonical} DR-plan and showing a strong Church-Rosser property: \vemph{all canonical DR-plans for isostatic or underconstrained 2D qusecs are optimal}.

% Also in Section \ref{sec:DRP}, we give an efficient (\candrpcomplexity) algorithm to find a canonical (and hence optimal) DR-plan for all 3 types of 2D qusecs mentioned above (Sections \ref{sec:DRP}, \ref{sec:bodypin}, and \ref{sec:pinnedline}).
% The canonical DR-plan elucidates the essence of the NP-hardness of finding optimal DR-plans for over-constrained systems.
% Furthermore, our optimal/canonical DR-plan satisfies desirable properties such as the previously studied Cluster Minimality \cite{hoffman2001decompositionI} (see Figure~\ref{fig:demo_graph:clustmindrp}).

In Section \ref{sec:DRP}, we define a so-called \dfn{canonical DR-plan} and prove a strong Church-Rosser property: all canonical DR-plans for isostatic or underconstrained 2D qusecs are optimal. In so doing, we navigate the NP-hardness barrier present in the general form of the DR-planning problem; the canonical DR-plan elucidates the essence of the NP-hardness of finding optimal DR-plans for over-constrained systems. Furthermore, our optimal/canonical DR-plan satisfies desirable properties such as the previously studied Cluster Minimality \cite{hoffman2001decompositionI} (see Figure~\ref{fig:demo_graph:clustmindrp}). Also in this section, a polynomial time (\candrpcomplexity) algorithm is provided to find a canonical DR-plan for isostatic bar-joint graphs.

% that such a plan will be optimal, in the sense that it minimizes the fan-in over all nodes, for isostatic bar-joint graphs.
% Additionally, a polynomial time (\candrpcomplexity) algorithm is provided to find a canonical DR-plan for isostatic bar-joint graphs.


In Section \ref{sec:recomb}, we give a method to deal with the algebraic complexity of recombining the realizations or solutions of child subsystems into a solution of the parent system \cite{sitharam2010optimized,sitharam2006well,sitharam2010reconciling}. Specifically, we define the problem of minimally modifying the indecomposable recombination system so that it becomes decomposable via a small DR-plan and yet preserves the original solutions in an efficiently searchable manner. In Section \ref{sec:table}, we show formal connection to well known problems such as optimal completion of underconstrained systems \cite{joan-arinyo2003transforming,sitharam2005combinatorial,gao2006ctree} and to find paths within the connected components. When the modifications are bounded, we obtain new, efficient algorithms for realizing both isostatic and underconstrained qusecs by leveraging recent results about Cayley parameters in \cite{sitharam2010convex,sitharam2011cayleyI,sitharam2011cayleyII} (see Sections \ref{sec:2-tree-reduction} and \ref{sec:tdecomp}).

\todo{Section \ref{sec:recomb} addresses the issue of large, indecomposable subgraphs in the optimal DR-plan of bar-joint graphs by proposing a novel method of modification. By dropping edges to get a convex configuration space, realizations of the original linkage can be found. Also problem relations...}

While both of these sections focus on bar-joint graphs, the theory is easily extended to other qusecs. The following sections focus on explicit applications by modeling rigidity in certain materials as qusecs.

In Section \ref{sec:bodypin} and \ref{sec:pinnedline}, we briefly describe applications of the above techniques to modeling, analyzing, and designing specific properties in 2D material layers~\cite{Jackson2008bodypin}. For Examples \ref{materialexample4} and \ref{materialexample5}, we discuss boundary-conditions for achieving various desired properties of body-hyperpin systems. For Example \ref{materialexample3}, we discuss canonical and optimal DR-plans for pinned line incidence systems \cite{sitharam2014incidence}).

We intend to make software implementation and videos available upon request and publicly available for the final version of the paper.












\subsection{The Importance of an Optimal DR-plan}
% \subsection{Importance of an Optimal DR-plan}
% The use of a DR-plan in designing especially self-similar qusecs is immediately evident, shown in Figures \ref{fig:c2c3ofk33s} and \ref{fig:bodypindrp}.

% In fact, knowing an optimal DR-plan is crucial for design and analysis of even non-self-similar qusecs and their realizations with various desirable properties.


Furthermore, recursive block decomposition of the rigidity matrix, stress space, and external stresses can be obtained using the DR-plan. Each block is the rigidity matrix of an isostatic subgraph. The blocks intersect on isostatic or trivial subgraphs. This recursive block decomposition of the rigidity matrix $R$ corresponds to a recursive decomposition of the stress vectors $s$ that balance a given external stress $t$ since $sR = t$. Moreover, the external stress vector $t$ is recursively ``distributed'' into external stresses for each node of the DR-plan. When the rigidity matrix and stress vectors contain indeterminates, this can be used to design underlying graph and parameters $\delta$ to obtain an optimal distribution of stresses on the geometric primitives. When the realization of the system \nameit is given, this can be used to analyze the distribution of a given external load.

