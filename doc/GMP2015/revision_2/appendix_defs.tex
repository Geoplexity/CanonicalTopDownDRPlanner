\section{Basic Definitions and Theory}
\label{sec:appendix:defs}

In this paper, a \dfn{geometric constraint system} is a multivariate polynomial (usually bilinear or quadratic) system $G(x,\delta)=0$, representing constraints with parameters $\delta$ between geometric primitives  in $\mathbb{R}^2$ represented collectively as $x\in \mathbb{R}^n$.
%
When the type of constraint (system) is fixed, the system is simply represented as $(G,\delta)$, where $G$ is the underlying constraint (hyper)graph $G = (V,E)$ with the vertices $V$ representing the geometric primitives in $\mathbb{R}^2$ and (hyper)edges $E$ representing the constraints, each with an associated parameter $\delta$.
%
For example, a \dfn{bar-joint system or linkage} $(G,\delta)$, is a graph $G=(V,E)$ with fixed length bars as edges, i.e. $\delta: E \rightarrow \mathbb{R}$; this represents the distance constraint system $\| x_u -x_v \|_2 = \delta_{u,v}$ for  $(u,v) \in E$, where $x_u \in \mathbb{R}^2$ represents the coordinates of $u\in V$.

% \noindent
% \note Geometric constraint systems can also have inequalities in addition to equations, where the parameters in $\delta$ are small intervals of values rather than exact values.

\sidenote{Geometric constraint systems can also have inequalities in addition to equations, where the parameters in $\delta$ are small intervals of values rather than exact values.}

In this paper, we will consider 3 types of constraint systems: bar-joint as well as \dfn{body-hyperpin} (defined formally in Section \ref{sec:bodypin}) and \dfn{pinned line-incidence} (defined formally in Section \ref{sec:pinnedline}). In all of these cases, a Cartesian \dfn{realization} or \dfn{solution} $G(p)$ of $(G,\delta)$ is an assignment of coordinates or Euclidean transformations (poses), $p: V \rightarrow \mathbb{R}^2$ or $\mathbb{R}^3$, to the vertices of $G$ satisfying the constraints with parameters $\delta$, modulo orientation preserving isometries (Euclidean rigid body motions for distance constraint).

Although the realization space itself depends on the constraint parameters $\delta$, many relevant \dfn{generic} properties of the constraint system $G(x,\delta)$ are properties of the constraint (hyper)graph $G$ and do not depend on $\delta$ (or they hold for all but a measure zero set of $\delta$ values). Many of these are properties of the Jacobian $\Delta_x G(x,\delta)$, often called the appropriate \dfn{rigidity matrix of $G$} (a matrix of indeterminates). For example, the \dfn{bar-joint rigidity matrix of the graph $G = (V,E)$} is a matrix of indeterminates representing the Jacobian of the distance map $\| x_u -x_v \|_2$  for $(u,v) \in E$. The matrix has $2$ columns per vertex in $V$ and one row per edge in $E$, where the row corresponding to edge $(u,v)$ contains the 2 coordinate indeterminates for $x_u -x_v$ (resp. $x_v-x_u$) in the 2 columns for $u$ (resp.\ $v$), that is 4 non-zero entries per row.


% XXXXXXXXX
% something has to be said along with asimow roth
% (infinitesimal rigidity implies rigidity) When the rigidity matrix has
% appropriate rank, the realizations or solutions of the corresponding
% constraint system are generically isolated and zero-dimensional if the
% constraint system has equalities and exact values for the parameters
% $\delta$. If the constraint system has inequalities, i.e, if the
% $\delta$ lie in an interval, this claim is approximate: the solutions
% are isolated small, full-dimensional  connected components.

One important property of generic constraint system or (hyper)graph
%that depends on the appropriate rigidity matrix of indeterminates
is \dfn{rigidity} (\note we refer to these as properties of the constraint system or as properties of the underlying (hyper)graph interchangeably), i.e., the realizations or solutions of the corresponding constraint system being generically isolated and zero-dimensional.%(if the constraint system has equalities and exact values for the parameters $\delta$).
The result by Asimow and Roth \cite{asimow1978rigidity}  shows a constraint (hyper)graph is rigid if it is generically \dfn{infinitesimally rigid}, i.e.\ the number of independent rows of its appropriate rigidity matrix is at least the number of columns less the number of rigid body motions, which is 3 for bar-joint systems.
%This condition is called generic {\em infinitesimal rigidity}.

% \noindent
% \note If the constraint system has inequalities, i.e, if the $\delta$ lie in an interval, the definition of rigidity is approximate: the solutions are isolated small, full-dimensional  connected components.

\sidenote{If the constraint system has inequalities, i.e, if the $\delta$ lie in an interval, the definition of rigidity is approximate: the solutions are isolated small, full-dimensional  connected components.}
% generic rigidity, i.e.\ there exist at most finitely many solutions to the system at generic points,  %to the algebraic system

Other generic constraint system or (hyper)graph properties are the following:
% (\note we refer to these as properties of the constraint
% system or as properties of the underlying (hyper)graph
% interchangeably).
A constraint (hyper)graph $G$ is \dfn{independent} if its appropriate rigidity matrix of indeterminates has independent rows (i.e, the determinant of some square submatrix is not identically zero).
%It is {\em rigid} if the number of independent rows of the rigidity matrix is
%at least the number of columns less the number of rigid body motions,
%which is 3 for distance constraint systems.
It is \dfn{isostatic (minimally rigid, wellconstrained)} if it is both rigid and independent.
%if the number of generically independent rows or the rank of the appropriate rigidity matrix is maximal.
%For example, the maximal rank of a bar-joint rigidity matrix is $2|V| - 3$,
%where $3$ is the number of rotational and translational degrees-of-freedom of a rigid body in $\mathbb{R}^2$.
%The graph $G$ is {\em rigid} if there exists some spanning subgraph $S\subseteq G$ such that $S$ is wellconstrained.
It is \dfn{flexible} if it is not rigid, \dfn{underconstrained} if it is independent and not rigid, or \dfn{overconstrained} if it is not independent.

Defining the combinatorial independence of a subset of edges $E'\subseteq E$ to be the generic independence of corresponding rows in the appropriate rigidity matrix of indeterminates, we obtain the \dfn{rigidity matroid} of a constraint (hyper)graph $G = (V,E)$.
%The 2-dimensional {\em rigidity matroid} of a constraint (hyper)graph $G = (V,E)$ is a linear matroid  on $E$,
%where a subset of edges $E' \subseteq E$ is {\em independent} in the matroid,
%if the set of corresponding rows in the appropriate rigidity matrix of indeterminates are linearly independent.
There are various results on combinatorial characterization of independence and rigidity  and rigidity matroids for different types of graphs. For bar-joint rigidity matroid, the famous Laman's theorem \cite{laman1970graphs} states that the underlying graph is isostatic if and only if $|E| = 2|V|-3$ and $|E'| \le 2|V'|-3$ for every induced subgraph with at least 2 vertices. The result by Lovasz and Yemini \cite{lovasz1982generic}  shows that all 6-vertex-connected graphs are rigid in the plane. For bar-body rigidity matroid, Tay \cite{tay1976rigidity} proved that the underlying multigraph is isostatic if and only if it can be decomposed as $3$ edge disjoint spanning trees. White and Whiteley \cite{white1987algebraic} gave the same characterization using a different technique to study the algebraic-geometric conditions of genericity, called pure condition. Lee, Streinu and Theran \cite{lee2007graded} defined the \dfn{$(k,l)$-sparsity matroid}, where a hypergraph $G$ is called \dfn{$(k,l)$-sparse} if $|E'| \le k|V'| - l$ for any induced subgraph $(V',E')$ with at least 2 vertices, and \dfn{$(k,l)$-tight} if is $(k,l)$-sparse and $|E| = k|V| - l$. In general, given a $d$-uniform hypergraph, a $(k,l)$-sparsity condition is matroidal as long as $l \le dk-1$.



% In this paper, a \dfn{qusecs} is any \dfn{independent} geometric constraint system of one of the 3 types mentioned above.

% % \medskip\noindent
% % \note In the remainder of this section and sections~\ref{sec:DRP} and \ref{sec:recomb} we only consider bar-joint qusecs and graphs. Relevant formal analogies for the other 2 types of qusecs and (hyper)graphs are given in the subsequent 2 sections on Applications.

% \sidenote{In the remainder of this section and Sections~\ref{sec:DRP} and \ref{sec:recomb} we only consider bar-joint qusecs and graphs. Relevant formal analogies for the other 2 types of qusecs and (hyper)graphs are given in the subsequent 2 sections on Applications (\ref{sec:bodypin} and \ref{sec:pinnedline}).}



%XXXNow comes the definition of stress vector and flex vector.
%NOTE: something has to be said about what flexes and stresses mean


%TODO
Given a bar-joint graph, a vector (of indeterminates) in the right null space of its rigidity matrix is called a \dfn{flex vector}. It has 2 entries per vertex and represents the internal motions of the system (i.e.\ modulo rigid body motions). For rigid graphs, all flex vectors are identically zero.
%
A vector (of indeterminates) in the left null space of the rigidity matrix is called a \dfn{stress vector}. It has one entry per edge and represents a \dfn{self stress} of the system. For independent graphs, all stress vectors are identically zero.
%
An \dfn{external stress} $t$ is a $1\times 2|V|$ vector  (of indeterminates) which specifies a 2D vector acting at each vertex such that the stress balance equation $sR = t$ holds, where $s$ is a vector of internal stresses, one for each bar, and $R$ is the rigidity matrix.
