\section{Conclusion}
\label{sec:conclusion}

We have clarified the main source of complexity for the optimal DR-plan and recombination problems. For the former problem, when there are no overconstraints (as is the case for 2D qusecs whose realizations are many common types of layered materials), we defined a canonical DR-plan and showed that any canonical DR-plan is guaranteed to be optimal, a strong Church-Rosser property. This gives an efficient (\candrpcomplexity) algorithm to find an optimal DR-plan that satisfies other desirable characteristics.

We have also described a novel method of efficiently realizing a 2D qusecs from the optimal DR-plan by modifying the otherwise indecomposable systems  at nodes of a DR-plan. These results rely on a recent theory of convex Cayley configuration spaces. Relationships and reductions between these and previously studied problems were formally clarified.

We then modeled specific layered materials using extensions of the above theoretical results including the motivating Examples 1-5 in the introduction.


% \section{Open Problems}
\subsection{Open Problems}
\label{sec:appendix:b}
\label{sec:futurework}
\label{sec:open}

The first set of problems are from Section \ref{sec:DRP}:
\begin{openproblem}
    Is there a more efficient algorithm than \candrpcomplexityv\ to find the canonical DR-plan of isostatic 2D bar-joint graphs?
\end{openproblem}

\begin{conjecture}
\label{conj:mfaisoptimal}
    The Modified \frontier\ Algorithm (MFA)~\cite{lomonosov2004graph} finds a canonical, and hence optimal, DR-plan.
\end{conjecture}

The difficulty of proving Conjecture~\ref{conj:mfaisoptimal} arises from the fact that MFA, although running in time $O(n^3)$, is a bottom-up algorithm, involving complex datastructures. However, a proof of optimality, even if it exists, would not be possible without the new notion of a canonical DR-plan at hand. The intuition for this conjecture comes from the similarity of the DR-plan generated by MFA to that of the sequential decomposition described in the proof of Theorem~\ref{theorem:algo_complexity}.  Since it is known \cite{lomonosov2004graph} that the DR-plan generated by MFA is cluster-minimal, an alternate conjecture is the following.

\begin{conjecture}
\label{conj:mfaisoptimal:rephrase}
For independent graphs, cluster-minimal DR-plans are optimal. In fact, for independent graphs, cluster-minimality and canonical are equivalent properties of a DR-plan.
\end{conjecture}


\begin{openproblem}
\label{open:sparsitymatroid}
Although generic rigidity is a property of graphs, and moreover,  in the case of qusecs, generic rigidity has a combinatorial sparsity and tightness-based characterization, the original definition of independence in the rigidity matroid requires an algebraic notion of independence of vectors of indeterminates over $\RR$. Thus the definition of the DR-plan requires algebra over the reals.
In fact, the recursive decomposition problem is not tied to geometric constraint graphs or an algebraic-geometric or mechanical notion of rigidity, and can be defined for any graph using the notion of an abstract rigidity matroid~\cite{graver93book}. This is a type of matroid with two additional matroid axioms; abstract rigidity matroids can be defined in a purely graph-theoretic manner,  with no need for algebra in their definition. However, such abstract rigidity need not have a sparsity characterization. On the other hand, there are sparsity matroids that do not correspond to any notion of abstract rigidity.
However, when an abstract rigidity matroid is also a sparsity matroid, then the techniques of this paper directly apply and we can obtain purely combinatorially defined recursive decompositions of graphs.
\end{openproblem}


A few natural  open questions concern the following common theme that runs through the optimal recombination and later sections of the paper:

\begin{openproblem}
    For fixed $k$, we have
    %
    polynomial time optimal DR-planning (Section~\ref{sec:DRP}),
    %
    recombination (modification) in the presence of $k$ overconstraints,
    %
    optimal modification for decomposition OMD$_k(G)$ when at most $k$ constraints are removed (Section~\ref{sec:recomb}),
    %
    and also optimal completion using at most $k\le 2$ constraints in the body-pin and triangle-multipin cases for a somewhat different optimization of the DR-plan (Section~\ref{sec:table}).
    %
    However, in the running time of all of these algorithms, $k$ appears in the exponent. Can $k$ be removed from the exponent?
\end{openproblem}

One problem in the above theme is from Section \ref{sec:bodypin}.
\begin{openproblem}
    What is the complexity of the optimal completion problem when the given graph has more than 2-dofs? Our proof for the 1 and 2-dof cases relied heavily on the matroidal properties of their corresponding $(k,l)$-tightness. For higher number of dofs, the $(k,l)$ characterization is no longer matroidal \cite{Lee:2007:PGA}. As a result, the major obstacle is that there is no easy way of obtaining an optimal or canonical $k$-dof DR-plan in general. Even assuming such  a DR-plan is available, if higher dofs had the same characteristics, Observation \ref{obs:algebraic_completion} raises questions about the correct measure of DR-plan size that captures algebraic complexity for recombining graphs with many dofs (this is not an issue in the isostatic case). Unless some restrictions can be found and taken advantage of, the $k$-dof optimal completion problem would  have complexity exponential in $k$.
\end{openproblem}

Another problem from the above theme is from Section \ref{sec:recomb}
\begin{openproblem}
    What is the complexity of the restricted OMD (optimal modification for decomposition) problem? This has the potential to be difficult. For example, when the isostatic completion is required to be a 2-tree the restricted OMD problem is reducible to the maximum spanning series-parallel subgraph problem shown by \cite{cai1993spanning} to be NP-complete even if the input graph is planar of maximum degree at most 6. However, since the OMD problem has other input restrictions such as not having any proper isostatic subgraphs, it is not clear if the reverse reduction exists and hence it is unclear whether the OMD problem is NP-complete.

    The same holds for the restricted OMD problem where the isostatic completion is required to be a tree-decomposable graph of low Cayley complexity (i.e.\ have special, small DR-plans). One potential obstacle to an indecomposable graph $G$'s membership in the restricted OMD$_k$ for small $k$ is if $G$ is tri-connected and has large girth. In fact, 6-connected (hence rigid) graphs with arbitrarily large girth have been constructed in \cite{servatius2000rigidity}.
\end{openproblem}

The next is the reverse direction of Observation \ref{obs:OC_to_OMD} in Section \ref{sec:table}.
\begin{openproblem}
    Is the OMD (optimal modification for decomposition) problem reducible to the OC (optimal completion) problem?
\end{openproblem}

More general problem directions are the following.
\begin{openproblem}
    Combinatorial rigidity for periodic structures is an active area of research. This paper motivates a study of the rigidity of self-similar structures, with self-similar groups replacing periodic groups.
\end{openproblem}

\begin{openproblem}
    The pinned line incidence structures of Section \ref{sec:pinnedline}, for example in the case of collagen microfibrils, whose function is elastic contraction, should be considered congruent under projective transformations. I.e.\ the projective group should be factored out as a trivial motion in a new project for extending the combinatorial rigidity characterization of such systems. (Currently we permit no trivial motions at all).
\end{openproblem}

\begin{openproblem}
    Experimental validation (either computationally or physical experiments) of predications based on the material model and theory used in this paper. This can be done by modeling known materials and putting stresses on them, seeing if the prediction is observed in the real material. Or, our theory could be used to design new materials, which can be tested to see if they possess predicted properties.
\end{openproblem}

