\section{Introduction}
Modeling, analysis and design of quasi-uniform or self-similar layered materials using geometric constraint systems
 

Some properties of natural and engineered materials can be studied by treating them as stacked two dimensional (2D) layers, 
where the structure within each layer may be recursively 
self-similar \cite{Intro1}, 
spanning multiple scales. Within any one scale, the material layer is generally aperiodic, and quasi-uniform,
consisting of a few repeated motifs appearing in disordered arrangments.
Each layer is not necessarily planar, i.e., it consists of multiple, inter-constraining planar (genus 0) monolayers. 
Furthermore, a layer is often   either isostatic or somewhat flexible, and rarely self-stressed, at any scale. These properties are
consistent with a self-assembled 2D structure that minimizes mass, optimally distributes external stresses and itself 
participates in the assembly of diverse and multifunctional, 
larger 2D structures.

Examples include:

\begin{itemize}
    
    \item (cross-section) of microtubules \cite{Necklace1} e.g., in ciliary membranes and transitions \cite{Necklace2}
    
    \item (cross-section) of hierarchical bone structure \cite{XX}
    
    \item interlinked cellulose or collagen microfibril monolayers e.g., in cell-walls \cite{CellWalls1} \cite{CellWalls1}
    
    \item more recent, engineered examples including  disordered graphene layers \cite{Graphene1} \cite{Graphene2} sometimes reinforced 
    by  microfibrils; and DNA assemblies \cite{Microfibrils1} including a recent Sierpinski carpet, bringing other structures 
    \cite{Microfibrils2} within reach.
    
    \item  silica bi-layers, glass \cite{SilicaGlass1} \cite{SilicaGlass2}, and materials that behave like assemblies of 
    2D particles under non-overlap constraints, e.g, jammed 
    disks on the plane \cite{JammedDisk1}

   
\end{itemize}


In order to study structural and mechanical properties such as rigidity, flexes and distribution of external stresses 
within the material layer, it is natural to model a material layer as a solution of geometric constraint systems 
arising from appropriate geometric primitives under metric or algebraic constraints. 
Such  2D {\em qusecs} (quasi-uniform or self-similar constraint systems) can be used to 
to understand or design material layers (their solutions) with desired properties.

\subsection{Previous Work}
We now briefly survey existing techniques for studying 2D {\em qusecs,}  many of which are
{\it bar-joint} systems (Examples 1,2 above), {\it multibody-pin} systems (Example 4,5) or {\it pinned-line incidence} systems (Example 3). 
The limitations of these techniques directly motivate the contributions of this paper.

Fast, graph-based algorithms exist (pebble-game \cite{XX}),   
for locating all maximal, generically rigid subsystems \cite{XX}. 
When the input is rigid, these algorithms do nothing, i.e, the output is the same as the input.
However, both for self-similar or just aperiodic 2D qusecs, it is imperative to recursively decompose rigid systems 
into their rigid subsystems, down to the level of geometric primitives, in order to 
understand or design properties at all scales, such as: rigidity, flexes, distribution 
of external stresses as well as robustness/sensitivity to constraint variations.
 
Recursive decomposition of geometric constraint systems has been formalized \cite{XX} and well-studied \cite{XX} 
as the Decomposition-Recombination (DR)-planning problem.
For the above-mentioned classes of 2D qusecs, generic rigidity is 
a combinatorial property and hence decomposition can, in principle, be done by a graph-based algorithm, without
involving geometric information in the constraint system.  
However, since many  such decompositions can exist for a given constraint system, 
criteria defining desirable or optimal DR-plans and DR-planning algorithms were given in \cite{XX}. An {\em optimal DR-plan} is one
that minimizes the maximum number of child subsystems of any parent system, a property that is 
crucial for obtaining a solution to the parent constraint system from the solutions of the child systems.

For a special class of 2D qusecs, namely  {\em tree-decomposable} systems \cite{XX} common in computer aided mechanical design, 
(which includes ruler-and-compass and Henneberg-I constuctible systems), all DR-plans turn out to 
be optimal. This satisfies the so-called {\em Church-Rosser} criterion, leading to highly efficient DR-planning algorithms.
However, for general 2D qusecs, even when restricted to bar-joint systems, 
the optimal DR-planning problem was shown to be NP-hard \cite{XX}.
Subsequently, alternate criteria were suggested   
such as {\em cluster minimality} 
requiring parent systems to be composed of  
a minimal set of at least 2 rigid proper subsystems (i.e., no proper subset forms a rigid system); 
and {\em proper maximality}, requiring child subsystems
to be maximal rigid proper subsystems of the parent system.
While polynomial time algorithms were given \cite{XX} to generate DR-plans meeting the cluster minimality criterion,
no such algorithm is known for the latter criterion.

For fairly general constraint systems, there are algorithms that use a constraint system's DR-plan \cite{XX} for solving the system.
They optimize the algebraic complexity of (re)combining child system solutions, 
and algebraically solve the optimized recombination system to give a solution to the parent system at any
given level of the DR-plan. However, the generality of these algorithms trades-off with efficiency. 
Faster algorithms should exist for 2D cusecs that leverage recent developments on so-called Cayley configuration spaces
\cite{XX,XX}.

For underconstrained 2D cusecs obtained from various subclasses of tree-decomposable systems, algorithms have been developed \cite{XX} that 
leverage Cayley configuration space theory to provide descriptions of connected components and find paths within the connected components 
of the standard cartesian configuration spaces. However, no such algorithms are known for non-tree-decomposable systems.

 
  


\subsection{Contributions}
In this paper, we  navigate the NP-hardness barrier for finding optimal DR-plans by defining a 
so-called {\it canonical} DR-plan and showing that all canonical DR-plans for isostatic 2DGMCS are optimal. 
We then give an efficient ($O(n^2)$) algorithm to find a canonical DR-plan for all 2DGMCS. The notion of a canonical DR-plan 
illuminates the essence of the NP-hardness of finding optimal DR-plans for overconstrained 2DGMCS and gives classes of 
overconstrained 2DGMCS for which the above  algorithm finds an optimal DR-plan. We then show how to judiciously combine 
existing methods for recombination, search of so-called, convex Cayley configuration spaces, and navigation of realizations, 
to explore a connected component of the configuration space of a flexible 2DGMCS, given a starting configuration. 
Software implementation can be found at: \todo{Provide Link}.

The starting point of this paper is the observation that the above 2D Geometric Materials Constraint Systems (resp. 2D-GEMS) 
belong to a  

However, these systems do not 
belong in the largest combinatorial class of 2D systems, whose configuration spaces have formally proven  characterizations and 
algorithmic guarantees (e.g. for algebraic complexity, path and volume measures, and  topology).

Specifically, these results \todo{[Citation Needed]} rely on the fact that rigid systems in this  class, in contrast to GMCS, 
have a straightforward and immediate {\sl optimal} recursive decomposition, (so-called {\sl optimal} DR-plan) and  recombination 
or realization (with a formal disambiguation by minimal orientation specification).

It is known \todo{[Citation Needed]} that already the optimal DR-planning problem for general 2DGMCS is NP-hard, intuitively 
because of a combinatorial explosion of polynomially-sized DR-plans.  


\subsection{Previous work}

Computer aided mechanical design represents a mature application of geometric constraint systems,  leading to and borrowing 
from deep and fundamental problems and techniques in distance geometry, convex analysis, combinatorics, algebra, topology 
and complexity theory.

Broadly, these problems can be classified into three types that draw upon each other, and possess mathematical, algorithmic, 
and software implementation aspects: (i) combinatorial (generic) minimal rigidity (or isostaticity or well-constrained-ness) 
characterization, (ii) (optimal) recursive decomposition (DR plans), recombination and realization, (iii) configuration space 
characterization (topology, complexity, path and volume measures).

In general, proofs of combinatorial (generic) minimal rigidity characterizations for geometric constraint systems rely on 
underlying matroids \todo{[Citation Needed]} In these cases, more-or-less greedy algorithms such as simplified network-flow, 
or so-called pebble games can be used.

Nothing nontrivial known until now for optimal DR plans, although efficient algorithms are known for DR-plans satisfying 
various other formal properties.

Recombination and realization navigation of realizations (disambiguation with orientation, faster with overconstraints, mention 
also totally different distance geometry method - low rank PSD matrix completions SDP/SVD)

Gelfand/McPherson, Walker's conjecture solution, Convex Cayley, EASAL and Caymos for low Cayley complexity and bijective 
representation using global rigidity.
