\section{Introduction}
Modeling, analysis and design of quasi-uniform or self-similar material layers using geometric constraint systems
 

Many natural and engineered materials are stacked two dimensional (2D) layers, where the structure within each layer may be recursively 
self-similar \cite{Intro1}, 
spanning multiple scales. Within any one scale, the material layer is generally aperiodic, and quasi-uniform,
consisting of a few repeated motifs appearing in disordered arrangments.
Each layer is not necessarily planar, i.e., it consists of multiple, inter-constraining planar (genus 0) monolayers. 
Furthermore, a layer is often   either isostatic or somewhat flexible, and rarely self-stressed, at any scale. These properties are
consistent with a self-assembled 2D structure that minimizes mass, optimally distributes external stresses and itself 
participates in the assembly of diverse and multifunctional, 
larger 2D structures.

Examples include:

\begin{itemize}
    
    \item (cross-section) of microtubules \cite{Necklace1} e.g., in ciliary membranes and transitions \cite{Necklace2}
    
    \item (cross-section) of hierarchical bone structure \cite{XX}
    
    \item interlinked cellulose or collagen microfibril monolayers e.g., in cell-walls \cite{CellWalls1} \cite{CellWalls1}
    
    \item more recent, engineered examples including  disordered graphene layers \cite{Graphene1} \cite{Graphene2} sometimes reinforced 
    by  microfibrils; and DNA assemblies \cite{Microfibrils1} including a recent Sierpinski carpet, bringing other structures 
    \cite{Microfibrils2} within reach.
    
    \item  silica bi-layers, glass \cite{SilicaGlass1} \cite{SilicaGlass2}, and materials that behave like assemblies of 
    2D particles under non-overlap constraints, e.g, jammed 
    disks on the plane \cite{JammedDisk1}

   
\end{itemize}


In order to study structural and mechanical properties such as rigidity, flexes and distribution of external stresses 
within the material layer, it is natural to model a material layer as a solution of geometric constraint systems 
arising from appropriate geometric primitives under metric or algebraic constraints. 
Such  2D {\em qusecs} (quasi-uniform or self-similar constraint systems) can be used to 
to understand or design material layers (their solutions) with desired properties.

We now briefly survey existing techniques and their limitations, for 2D {\em qusecs,} which are most naturally treated as
{\it bar-joint} systems (Examples 1,2 above), {\it multibody-pin} systems (Example 4,5) or {\it pinned-line incidence} systems (Example 3). 
The limitations directly motivate the contributions of this paper.

First, all of these have highly efficient combinatorial algorithms (pebble-game \cite{XX}, or equivalently, simplified network flow \cite{XX}), 
for locating maximal, generically rigid subsystems \cite{XX}. When the input is rigid, these algorithms do nothing, i.e, the output is the same as the input.
However, both for self-similar or just aperiodic 2D qusecs, recursive decomposition into rigid subsystems
is imperative for understanding or designing rigidity, flexes, distribution 
of external stresses as well as robustness/sensitivity to constraint variations -- at all scales. 
In particular, recursive decomposition of rigid systems down to the geometric primitives is necessary.

Second,  decomposition of geometric constraint systems has been formalized \cite{XX} and well-studied \cite{XX} 
as the Decomposition-Recombination (DR)
-planning problem.
For the types of 2D qusecs listed above, generic rigidity is 
a combinatorial property and hence decomposition does not require
knowledge of the geometric information in the constraint system. 
In particular, since many different such decompositions typically exist, 
some criteria  to ensure good or optimal DR-plans were given in \cite{XX}. An optimal DR-plan is one
that minimizes the maximum number of child subsystems, over all its parent systems, a property that is 
crucial for obtaining a solution to the parent constraint system from the solutions of the child systems.
For special 2D qusecs such as so-called {\em tree-decomposable} systems, optimal DR-planning is immediate.

the optimal DR-planning problem was shown to be NP-hard for general 2D qusecs. 

Other criteria where any parent system is composed of  
a minimal set of maximal rigid proper subsystems  



Third, In addition, there are general algorithms for
recombination of  child system solutions, in order to obtain a solution to the parent system at any
given level of the DR-plan.


 Two key observations serve as the motivations for this paper:


\subsection{Contributions}

The starting point of this paper is the observation that the above 2D Geometric Materials Constraint Systems (resp. 2D-GEMS) 
belong to a  

However, these systems do not 
belong in the largest combinatorial class of 2D systems, whose configuration spaces have formally proven  characterizations and 
algorithmic guarantees (e.g. for algebraic complexity, path and volume measures, and  topology).

Specifically, these results \todo{[Citation Needed]} rely on the fact that rigid systems in this  class, in contrast to GMCS, 
have a straightforward and immediate {\sl optimal} recursive decomposition, (so-called {\sl optimal} DR-plan) and  recombination 
or realization (with a formal disambiguation by minimal orientation specification).

It is known \todo{[Citation Needed]} that already the optimal DR-planning problem for general 2DGMCS is NP-hard, intuitively 
because of a combinatorial explosion of polynomially-sized DR-plans.  In this paper, we  navigate this barrier by defining a 
so-called {\it canonical} DR-plan and showing that all canonical DR-plans for non-overconstrained 2DGMCS are optimal. 
We then give an efficient ($O(n^2)$) algorithm to find a canonical DR-plan for all 2DGMCS. The notion of a canonical DR-plan 
illuminates the essence of the NP-hardness of finding optimal DR-plans for overconstrained 2DGMCS and gives classes of 
overconstrained 2DGMCS for which the above  algorithm finds an optimal DR-plan. We then show how to judiciously combine 
existing methods for recombination, search of so-called, convex Cayley configuration spaces, and navigation of realizations, 
to explore a connected component of the configuration space of a flexible 2DGMCS, given a starting configuration. 
Software implementation can be found at: \todo{Provide Link}.


\subsection{Previous work}

Computer aided mechanical design represents a mature application of geometric constraint systems,  leading to and borrowing 
from deep and fundamental problems and techniques in distance geometry, convex analysis, combinatorics, algebra, topology 
and complexity theory.

Broadly, these problems can be classified into three types that draw upon each other, and possess mathematical, algorithmic, 
and software implementation aspects: (i) combinatorial (generic) minimal rigidity (or isostaticity or well-constrained-ness) 
characterization, (ii) (optimal) recursive decomposition (DR plans), recombination and realization, (iii) configuration space 
characterization (topology, complexity, path and volume measures).

In general, proofs of combinatorial (generic) minimal rigidity characterizations for geometric constraint systems rely on 
underlying matroids \todo{[Citation Needed]} In these cases, more-or-less greedy algorithms such as simplified network-flow, 
or so-called pebble games can be used.

Nothing nontrivial known until now for optimal DR plans, although efficient algorithms are known for DR-plans satisfying 
various other formal properties.

Recombination and realization navigation of realizations (disambiguation with orientation, faster with overconstraints, mention 
also totally different distance geometry method - low rank PSD matrix completions SDP/SVD)

Gelfand/McPherson, Walker's conjecture solution, Convex Cayley, EASAL and Caymos for low Cayley complexity and bijective 
representation using global rigidity.
