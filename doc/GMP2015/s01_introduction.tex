\section{Introduction}
\label{sec:prev}
Modeling, analysis and design of quasi-uniform or self-similar layered materials using geometric constraint systems
 
Some properties of natural and engineered materials can be studied by treating them as stacked two dimensional (2D) layers, 
where the structure within each layer may be recursively 
self-similar \cite{Intro1}, 
spanning multiple scales. Within any one scale, the material layer is generally aperiodic, and quasi-uniform,
consisting of a few repeated motifs appearing in disordered arrangments.
Each layer is not necessarily planar, i.e., it consists of multiple, inter-constraining planar (genus 0) monolayers. 
Furthermore, a layer is often  either {\sl isostatic or flexible, not self-stressed}, at any scale. These properties are
consistent with a self-assembled 2D structure that minimizes mass, optimally distributes external stresses and itself 
participates in the assembly of diverse and multifunctional, 
larger 2D structures.

Examples include:

\begin{enumerate}
  %  
    \item (cross-section) of microtubules \cite{Necklace1} e.g., in ciliary membranes and transitions \cite{Necklace2}
   % 
    \item (cross-section) of hierarchical bone structure \cite{XX}
    %
    \item interlinked cellulose or collagen microfibril monolayers e.g., in cell-walls \cite{CellWalls1} \cite{CellWalls1}
  %  
    \item more recent, engineered examples including  disordered graphene layers \cite{Graphene1} \cite{Graphene2} sometimes reinforced 
    by  microfibrils; and DNA assemblies \cite{Microfibrils1} including a recent Sierpinski carpet, bringing other structures 
    \cite{Microfibrils2} within reach.
    %
    \item  silica bi-layers, glass \cite{SilicaGlass1} \cite{SilicaGlass2}, and materials that behave like assemblies of 
    2D particles under non-overlap constraints, e.g, jammed 
    disks on the plane \cite{JammedDisk1}
%
\end{enumerate}
%
In order to study structural and mechanical properties such as rigidity, flexes and distribution of external stresses 
within the material layer, it is natural to model a material layer as a solution of geometric constraint systems 
arising from appropriate geometric primitives under metric or algebraic constraints. 
Such  2D {\bf{\em qusecs}} (quasi-uniform or self-similar constraint systems) can be used to 
to understand or design material layers (their solutions) with desired properties.
%
\subsection{Previous Work on Relevant 2D Geometric Constraint Systems}
We now briefly survey existing techniques for studying 2D {\em qusecs,}  many of which are
{\it bar-joint} systems (Examples 1,2 above), {\it multibody-pin} systems (Example 4,5) or {\it pinned-line incidence} systems (Example 3). 
The limitations of these techniques directly motivate the contributions of this paper.

\medskip\noindent\underbar{{\sl (i) Finding (vertex)-maximal generically rigid subsystems}}
Fast, graph-based algorithms exist (pebble-game \cite{XX}),   
for locating all maximal, generically rigid subsystems \cite{XX}. 
When the input is rigid, these algorithms do nothing, i.e, the output is the same as the input.
However, both for self-similar or just aperiodic 2D qusecs, it is imperative to recursively decompose rigid systems 
into their rigid subsystems, down to the level of geometric primitives, in order to 
understand or design properties at all scales, such as: rigidity, flexes, distribution 
of external stresses, boundary conditions for isostaticity, as well as behavior under constraint variations.
 
\medskip\noindent\underbar{{\sl (ii) Optimal Recursive Decomposition (DR-planning)}}
Recursive decomposition of geometric constraint systems has been formalized \cite{XX} and well-studied \cite{XX} 
as the {\sl Decomposition-Recombination (DR)-planning} problem.
For the above-mentioned classes of 2D qusecs, generic rigidity is 
a combinatorial property and hence each level of the decomposition should, in principle, be achievable by a graph-based algorithm as in (1), without
involving geometric information in the constraint system.  
Since many  such decompositions can exist for a given constraint system, 
criteria defining desirable or optimal DR-plans and DR-planning algorithms were given in \cite{XX}. An {\em optimal DR-plan} is one
that minimizes the maximum number of child subsystems of any parent system, a property that is 
crucial for obtaining a solution to the parent constraint system from the solutions of the child systems.
However, for general 2D qusecs, even when restricted to bar-joint systems, 
the optimal DR-planning problem was shown to be NP-hard \cite{XX}.

\medskip\noindent\underbar{{\sl (iii) DR-plans for special classes and with other criteria}}
For a special class of 2D qusecs, namely  {\em tree-decomposable} systems \cite{XX} common in computer aided mechanical design, 
(which includes ruler-and-compass and Henneberg-I constuctible systems), all DR-plans turn out to 
be optimal. This satisfies the so-called {\em Church-Rosser} criterion, leading to highly efficient DR-planning algorithms.
Subsequently, alternate criteria were suggested   
such as {\em cluster minimality} 
requiring parent systems to be composed of  
a minimal set of at least 2 rigid proper subsystems (i.e., no proper subset forms a rigid system); 
and {\em proper maximality}, requiring child subsystems
to be maximal rigid proper subsystems of the parent system.
While polynomial time algorithms were given \cite{XX} to generate DR-plans meeting the cluster minimality criterion,
no such algorithm is known for the latter criterion.

 
\medskip\noindent\underbar{{\sl (iv) Optimal Recombination and Solution Space Navigation}}
For fairly general 3D constraint systems, there are algorithms with formal guarantees that 
solve the parent (sub)systems that occur in the DR-plan \cite{XX, XX,XX}, 
provided the DR-plan meets some of the above criteria.
These algorithms  uncover underlying matroids to combinatorially optimize the algebraic complexity of (re)combining child system solutions, 
and algebraically solve the optimized recombination system to give a solution to the parent system at any
given level of the DR-plan. However, the generality of these algorithms trades-off against efficiency, and despite the optimization,
the best algorithms can still take
exponential time in the number of child subsystems in order to guarantee all solutions even for a single (sub)system in a DR-plan, 
and are prohibitively slow in practice. For the entire DR-plan, finding all solutions is barely tractable even with fast recombination algorithms
for the subsystems in the DR-plan. This is because even for the simplest systems, the problem is NP-hard \cite{XX} and there is a 
combinatorial explosion of solutions \cite{XX}.  Using the DR-plan and optimal recombination as a principled basis, high performance software exists \cite{XX,XX} to 
tame combinatorial explosion via user intervention.

 
\medskip\noindent\underbar{{\sl (v) Configuration Spaces of Underconstrained Systems}}
For underconstrained 2D bar-joint and multibody-pin cusecs obtained from various subclasses of tree-decomposable systems, algorithms have been developed \cite{XX, XX} that 
to complete them into isostatic systems \cite{XX, XX} and to find paths within the connected components \cite{XX,XX} of
standard cartesian configuration spaces. Most of the algorithms with formal guarantees leverage Cayley configuration space theory \cite{XX,XX}
to characterize subclasses of graphs and additional constraints that control combinatorial explosion, and provide faithful bijective representation 
of connected components and paths.
These algorithms have decreasing efficiency as the subclass of systems gets bigger, starting with
underlying partial 2-tree graphs (alternately, tree-width 2, series-parallel, $K_4$ minor avoiding),  moving on   
to 1 degree-of-freedom (dof) graphs with low Cayley complexity (which include common linkages such as the Strandbeest, Limacon and Cardioid), 
and further to general 1-dof tree-decomposable graphs. 
While software suites exist \cite{XX,XX}, no such formal algorithms and guarantees are known for non-tree-decomposable systems.
%
\subsection{Contributions}
\label{sec:cont}
The contributions of this paper are the following.
\begin{itemize}
\item
We  navigate the NP-hardness barrier mentioned in (2) above, for finding optimal DR-plans by defining a 
so-called {\it canonical} DR-plan and showing a strong Church-Rosser property: {\it all canonical DR-plans for isostatic or underconstrained (independent) 
2D qusecs are optimal}. 
\item
We give an efficient ($O(n^2)$) algorithm to find a canonical (and hence optimal) DR-plan for all 3 types of 2D qusecs mentioned above.
The canonical DR-plan  
illuminates the essence of the NP-hardness of finding optimal DR-plans for overconstrained systems, and characterizes those 
overconstrained 2D qusecs for which the above algorithm does find an optimal DR-plan. 
\item
We show how to judiciously combine 
existing methods for recombination with searching  convex or low complexity Cayley configuration spaces to  navigate the solution space of isostatic 2D qusecs, and
explore connected components of the configuration space of underconstrained 2D qusecs. 
\item
We briefly describe applications of the above techniques to modeling and analyzing specific 2D material layers mentioned above.
\item
Software implementation and videos can be found at: \todo{Provide Link}.
\end{itemize}

%%%%%%%
XXXXXXXXXXXXXXXXXXXXXXx

Computer aided mechanical design represents a mature application of geometric constraint systems,  leading to and borrowing 
from deep and fundamental problems and techniques in distance geometry, convex analysis, combinatorics, algebra, topology 
and complexity theory.

Broadly, these problems can be classified into three types that draw upon each other, and possess mathematical, algorithmic, 
and software implementation aspects: (i) combinatorial (generic) minimal rigidity (or isostaticity or well-constrained-ness) 
characterization, (ii) (optimal) recursive decomposition (DR plans), recombination and realization, (iii) configuration space 
characterization (topology, complexity, path and volume measures).

In general, proofs of combinatorial (generic) minimal rigidity characterizations for geometric constraint systems rely on 
underlying matroids \todo{[Citation Needed]} In these cases, more-or-less greedy algorithms such as simplified network-flow, 
or so-called pebble games can be used.

Nothing nontrivial known until now for optimal DR plans, although efficient algorithms are known for DR-plans satisfying 
various other formal properties.

Recombination and realization navigation of realizations (disambiguation with orientation, faster with overconstraints, mention 
also totally different distance geometry method - low rank PSD matrix completions SDP/SVD)

Gelfand/McPherson, Walker's conjecture solution, Convex Cayley, EASAL and Caymos for low Cayley complexity and bijective 
representation using global rigidity.
